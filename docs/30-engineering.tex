%% Методические указания к выполнению, оформлению и защите выпускной квалификационной работы бакалавра
%% 2.6 Технологический раздел
%%
%% Технологический раздел содержит обоснованный выбор средств программной реализации, описание основных (нетривиальных) моментов разработки и методики тестирования созданного программного обеспечения.
%%
%% В этом же разделе описывается информация, необходимая для сборки и запуска разработанного программного обеспечения, форматы входных, выходных и конфигурационных файлов (если такие имеются), а также интерфейс пользователя и руководство пользователя.
%%
%% Если для правильного функционирования разработанного программного обеспечения требуется некоторая инфраструктура (веб-приложение, база данных, серверное приложение), уместно представить её с помощью диаграммы развёртывания UML.
%%
%% Как уже говорилось, часть технологического раздела должна быть посвящена тестированию разработанного программного обеспечения.
%%
%% Модульное тестирование описывается в технологическом разделе.
%%
%% Системное тестирование может быть описано в технологическом или экспериментальном разделах, в зависимости от глубины его реализации и тематики бакалаврской работы.
%%
%% При проведении тестирования разработанного программного обеспечения следует широко использовать специализированные программные приложения: различные статические анализаторы кода (например, clang); для тестирования утечек памяти в языках программирования, где отсутствует автоматическая «сборка мусора», Valgrind, Doctor Memory и их аналоги, и т. п.
%%
%% Рекомендуемый объём технологического раздела 20—25 страниц.


\chapter{Технологический раздел}

\section{Структура баз данных}

На основе описанных выше сущностей для каждого из сервисов были разработаны спецификации таблиц баз данных, приведённых на рисунке \ref{img:erd-crow's-foot}.

\begin{table}[H]
	\caption{Спецификация таблицы User}
	\begin{tabular}{|l|l|l|}
		\hline
		\multicolumn{1}{|c|}{\textbf{Имя атрибута}} & \multicolumn{1}{c|}{\textbf{Тип атрибута}} & \multicolumn{1}{c|}{\textbf{Описание атрибута}} \\ \hline
		id                                          & UUID4                                      & Идентификатор пользователя                      \\ \hline
		email                                       & string                                     & Электронная почта                               \\ \hline
		password\_hash                              & string                                     & Хеш пароля                                      \\ \hline
		full\_name                                  & string                                     & ФИО                                             \\ \hline
		role                                        & string                                     & Роль \\ \hline
	\end{tabular}
\end{table}

\begin{table}[H]
	\caption{Спецификация таблицы Tag}
	\begin{tabular}{|l|l|l|}
		\hline
		\multicolumn{1}{|c|}{\textbf{Имя атрибута}} & \multicolumn{1}{c|}{\textbf{Тип атрибута}} & \multicolumn{1}{c|}{\textbf{Описание атрибута}} \\ \hline
		id                                          & UUID4                                      & Идентификатор тега                              \\ \hline
		name                                        & string                                     & Категория                                       \\ \hline
	\end{tabular}
\end{table}

\begin{table}[H]
	\caption{Спецификация таблицы Publication}
	\begin{tabular}{|l|l|l|}
		\hline
		\multicolumn{1}{|c|}{\textbf{Имя атрибута}} & \multicolumn{1}{c|}{\textbf{Тип атрибута}} & \multicolumn{1}{c|}{\textbf{Описание атрибута}} \\ \hline
		id                                          & UUID4                                      & Идентификатор публикации                        \\ \hline
		author\_id                                  & UUID4                                      & Идентификатор автора                            \\ \hline
		title                                       & string                                     & Заголовок                                       \\ \hline
		body                                        & string                                     & Тело                                            \\ \hline
		pub\_date                                   & timestamp                                  & Дата публикации                                 \\ \hline
		rating                                      & int                                        & Рейтинг                                         \\ \hline
	\end{tabular}
\end{table}

\begin{table}[H]
	\caption{Спецификация таблицы Comment}
	\begin{tabular}{|l|l|l|}
		\hline
		\multicolumn{1}{|c|}{\textbf{Имя атрибута}} & \multicolumn{1}{c|}{\textbf{Тип атрибута}} & \multicolumn{1}{c|}{\textbf{Описание атрибута}} \\ \hline
		id                                          & UUID4                                      & Идентификатор комментария                       \\ \hline
		author\_id                                  & UUID4                                      & Идентификатор автора                            \\ \hline
		publication\_id                             & UUID4                                      & Идентификатор публикации                        \\ \hline
		body                                        & string                                     & Тело                                            \\ \hline
		pub\_date                                   & timestamp                                  & Дата публикации комментария                     \\ \hline
		rating                                      & int                                        & Рейтинг                                         \\ \hline
	\end{tabular}
\end{table}

\begin{table}[H]
	\caption{Спецификация таблицы Subscription}
	\begin{tabular}{|l|l|l|}
		\hline
		\multicolumn{1}{|c|}{\textbf{Имя атрибута}} & \multicolumn{1}{c|}{\textbf{Тип атрибута}} & \multicolumn{1}{c|}{\textbf{Описание атрибута}} \\ \hline
		id                                          & UUID4                                      & Идентификатор сущности                          \\ \hline
		user\_id                                    & UUID4                                      & Идентификатор пользователя                      \\ \hline
		sub\_id                                     & UUID4                                      & Идентификатор объекта подписки                  \\ \hline
		type                                        & string                                     & Тип подписки (Тег или Автор)                    \\ \hline
	\end{tabular}
\end{table}

\begin{table}[H]
	\caption{Спецификация таблицы Publication\_tag}
	\begin{tabular}{|l|l|l|}
		\hline
		\multicolumn{1}{|c|}{\textbf{Имя атрибута}} & \multicolumn{1}{c|}{\textbf{Тип атрибута}} & \multicolumn{1}{c|}{\textbf{Описание атрибута}} \\ \hline
		id                                          & UUID4                                      & Идентификатор сущности                          \\ \hline
		publication\_id                             & UUID4                                      & Идентификатор публикации                        \\ \hline
		tag\_id                                     & UUID4                                      & Идентификатор тега                              \\ \hline
	\end{tabular}
\end{table}

\section{Сборка и деплой системы}

Для автоматической сборки проекта используется функциональность Github Actions \cite{github-actions}.
При отправке изменений в master ветку репозитория автоматически запускается сборка и тестирование каждого из микросервисов.
Конфигурация запуска автоматической сборки и тестирования
представлена на листинге \ref{lst:ci-cd}.

\begin{lstlisting}[
	caption={Конфигурация запуска автоматической сборки и тестирования},
	label={lst:ci-cd},
]
name: Django

on:
  push:
    branches: [master]
  pull_request:
    branches: [master]

jobs:
  build:
    runs-on: ubuntu-latest
    steps:
    - uses: actions/checkout@v3

    - name: Set up python
      uses: actions/setup-python@v4
      with:
        python-version: '3.10.4'

    - name: Install requirements
      run: python -m pip install -r requirements.txt

    - name: Migrate Session Service
      run: python backend_session/manage.py migrate

    - name: Migrate Publication Service
      run: python backend_publication/manage.py migrate

    - name: Migrate Subscription Service
      run: python backend_subscription/manage.py migrate

    - name: Migrate Statistics Service
      run: python backend_statistics/manage.py migrate

    - name: Test Gateway Service
      run: cd backend_gateway && python manage.py test

    - name: Test Session Service
      run: cd backend_session && python manage.py test

    - name: Test Publication Service
      run: cd backend_publication && python manage.py test

    - name: Test Subscription Service
      run: cd backend_subscription && python manage.py test

    - name: Test Statistics Service
      run: cd backend_statistics && python manage.py test
\end{lstlisting}

Для развертывания системы используется docker compose, каждый сервис и база данных каждого сервиса разворачиваются в отдельном контейнере.
Все контейнеры микросервисы связаны общей сетью, каждый микросервис связан отдельной сетью с контейнером с его базой данных.

\section{Деградация системы}

При отказе сервиса статистики должны продолжить полноценно работать все пользовательские сценарии за исключением просмотра статистики.

При отказе очередей для сбора статистики должны продолжить полноценно работать все пользовательские сценарии, одна статистика в это время собираться не будет.

При отказе сервиса подписок должны продолжить работать следующие пользовательские сценарии, кроме следующих:

\begin{itemize}
	\item подписка на автора;
	\item подписка на тэг;
	\item отписка от автора;
	\item отписка от тега;
	\item просмотр своей ленты публикаций по подпискам.
\end{itemize}


%\section{Выводы}
