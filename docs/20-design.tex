%% Методические указания к выполнению, оформлению и защите выпускной квалификационной работы бакалавра
%% 2.5 Конструкторский раздел
%%
%% В конструкторском разделе описывается разрабатываемый и/или модифицируемый метод или алгоритм.
%%
%% В случае если в бакалаврском проекте разрабатывается новый метод или алгоритм, необходимо подробно изложить их суть, привести всё необходимые для их реализации математические выкладки, обосновать последовательность этапов выполнения.
%% При этом для каждого этапа следует выделить необходимые исходные данные и получаемые результаты.
%%
%% При использовании известного алгоритма следует указать специфические особенности его практической реализации, присущие решаемой задаче, и пути их решения в ходе программирования.
%% Для описания метода или алгоритма необходимо выбрать наиболее подходящую форму записи (схема (ГОСТ 19.701-90), диаграмма деятельности, псевдокод и т. п.).
%% Учитывая, что на эффективность алгоритма непосредственно влияют используемые структуры данных, в данном разделе РПЗ целесообразно провести сравнительный анализ структур, которые могут быть применены в рамках программной реализации выбранного алгоритма, и обосновать выбор одной из них.
%% В конце описания разработанного и/или модифицируемого алгоритма должны быть приведены выбранные способы тестирования и сами тесты.
%%
%% Перед формированием тестовых наборов данных целесообразно указать выделенные классы эквивалентности.
%% В данной части расчётно-пояснительной записки могут также выполняться расчёты для определения объёмов памяти, необходимой для хранения данных, промежуточных и окончательных результатов работы программы, а также расчёты, позволяющие оценить время решения задачи на ЭВМ.
%% Эти результаты могут использоваться для обоснования правильности выбора метода и/или алгоритма из имеющихся альтернативных вариантов, а также для оценки возможности практически реализовать поставленную задачу на имеющейся технической базе.
%%
%% Другой важный момент, который должен найти своё отражение в конструкторском разделе, это описание структуры разрабатываемого программного обеспечения.
%% Обычно оно включает в себя:
%% — описание общей структуры — определение основных частей (компонентов) и их взаимосвязей по управлению и по данным;
%% — декомпозицию компонентов и построение структурных иерархий;
%% — проектирование компонентов.
%%
%% Для графического представления такого описания, если есть необходимость, следует использовать:
%% — функциональную модель IDEF0 с декомпозицией решения исходной задачи на несколько уровней (разрабатываемые модули обычно играют роль механизмов);
%% — спецификации компонентов (процессов);
%% — модель данных (ER-диаграмма);
%% — диаграмму классов;
%% — диаграмму компонентов;
%% — диаграмму переходов состояний (конечный автомат), характеризующих поведение системы во времени.
%%
%% Рекомендуемый объем конструкторского раздела 25—30 страниц.


\chapter{Конструкторский раздел}

\section{Архитектура системы}

В данной работе используется микросервисная архитектура, позволяющая разделить программную систему на небольшие, слабо связанные и легко изменяемые модули.

Было выделено 6 микросервисов.
\begin{itemize}
	\item Сервис пользовательского интерфейса. Данный сервис принимает запросы от пользователей и в зависимости от них выполняет запросы сервису-координатору и сервису авторизации для получения токен.
	\item Сервис-координатор должен реализовать диспетчеризацию запросов.
	\item Сервис публикаций — отвечает за добавление публикаций и хранение информации о них.
	\item Сервис подписок — осуществляет формирование персонального информационного окружения пользователя.
	\item Сервис регистрации и авторизации.
	\item Сервис статистики.
\end{itemize}

Каждый из микросервисов кроме сервиса пользовательского интерфейса и сервиса-координатора имеет свою базу данных и не обращается к базам данных других сервисов.

Топология разрабатываемой системы изображена на рисунке \ref{img:topology}.

\imgH{width=\linewidth}{topology}{Топология системы}

\section{Сущности системы}

На основе функциональных требований к выделенным подсистемам, а также объектов, о которых необходимо хранить данные в системе, была разработана схема данных приложения.
Результат её проектирования отображён на условной ER-диаграмме, представленной на рисунке \ref{img:erd-chen}.

\imght{width=\linewidth}{erd-chen}{ER-диаграмма данных системы}

На следующей стадии проектирования, добавив в схему данных атрибуты сущностей, получаем схему базы данных, которая изображена на рисунке \ref{img:erd-crow's-foot2}.

\imgH{width=\linewidth}{erd-crow's-foot2}{Схема базы данных системы}

\section{Взаимодействие систем}

Для описания поведения компонентов системы на единой оси времени используются диаграммы последовательности действий, при помощи которых можно описать последовательность действий для каждого прецедента, необходимую для
достижения цели.
Например, на рисунке \ref{img:sequence-diagram} изображён процесс получения списка публикаций пользователем на основе его подписок.

\imgH{width=\linewidth}{sequence-diagram}{Диаграмма последовательности действий при запросе пользователем публикаций на основе его подписок}

Сервис-координатор отправляет запрос сервису подписок на получение списка авторов и тегов, на которые подписан пользователь.
После получения данных сервис-координатор обращается к сервису публикаций, чтобы получить список публикаций по переданным авторам и тегам.
После окончания данной процедуры главный сервис ранжирует список по времени публикации, формирует веб-страницу и возвращает её пользователю.


%\section{Выводы}
