\documentclass{bmstu-gost-7-32}


%% ГОСТ 7.32-2017
%% 4 Структурные элементы отчёта
%%
%% Структурными элементами отчёта о НИР являются:
%%     — титульный лист;
%%     — список исполнителей;
%%     — реферат;
%%     — содержание;
%%     ~ термины и определения;
%%     ~ перечень сокращений и обозначений;
%%     — введение;
%%     — основная часть отчёта о НИР;
%%     — заключение;
%%     ~ список использованных источников;
%%     ~ приложения.
%%
%% Обязательные структурные элементы выделены полужирным шрифтом.
%% Остальные структурные элементы включают в отчёт о НИР по усмотрению исполнителя НИР с учётом требований разделов 5 и 6.


\begin{document}

\makecourseworktitle
	{Информатика, искусственный интеллект и системы управления} % Название факультета
	{Программное обеспечение ЭВМ и информационные технологии} % Название кафедры
	{Система ведения онлайн-дневников (блогов)} % Тема работы
	{ИУ7-23М} % Номер группы
	{Керимов~А.~Ш.} % ФИО студента
	{Щетинин~Г.~А.} % ФИО преподавателя
	{} % ФИО консультанта 1 (необязательный аргумент)
	{} % ФИО консультанта 2 (необязательный аргумент)

%\includepdf[pages=-]{inc/tz.pdf}
\setcounter{page}{3}
%%% ГОСТ 7.32-2017
%% 5.2 Список исполнителей
%%
%% 5.2.1 В список исполнителей должны быть включены фамилии и инициалы, должности, ученые степени, ученые звания и подписи руководителей НИР, ответственных исполнителей, исполнителей и соисполнителей, принимавших непосредственное участие в выполнении работы, с указанием их роли в подготовке отчета.
%%
%% 5.2.2 Если отчет выполнен одним исполнителем, его должность, ученую степень, ученое звание, фамилию и инициалы следует указывать на титульном листе отчета.
%% В этом случае структурный элемент отчета "СПИСОК ИСПОЛНИТЕЛЕЙ" не оформляют.
%%
%% 5.2.3 Основная задача нормоконтролера — проверка соблюдения норм и требований, установленных настоящим стандартом, соблюдение всех нормативных требований, соблюдения единообразия в оформлении структурных элементов и правил оформления отчета о НИР.
%%
%% 5.2.4 Список исполнителей следует оформлять в соответствии с 6.11.
%% Примеры оформления списка исполнителей к отчету о НИР приведены в приложении Б.


\StructuralElement{Список исполнителей}

%%% ГОСТ 7.32-2017
%% 5.3 Реферат
%%
%% 5.3.1 Общие требования к реферату отчета о НИР — по ГОСТ 7.9.
%%
%% 5.3.2 Реферат должен содержать:
%% — сведения об общем объеме отчета, количестве книг отчета, иллюстраций, таблиц, использованных источников, приложений;
%% — перечень ключевых слов;
%% — текст реферата.
%%
%% 5.3.2.1 Перечень ключевых слов должен включать от 5 до 15 слов или словосочетаний из текста отчета, которые в наибольшей мере характеризуют его содержание и обеспечивают возможность информационного поиска.
%%
%% 5.3.2.2 Текст реферата должен отражать:
%% — объект исследования или разработки;
%% — цель работы;
%% — методы или методологию проведения работы;
%% — результаты работы и их новизну;
%% — область применения результатов;
%% — рекомендации по внедрению или итоги внедрения результатов НИР;
%% — экономическую эффективность или значимость работы;
%% — прогнозные предположения о развитии объекта исследования.
%%
%% Если отчет не содержит сведений по какой-либо из перечисленных структурных частей реферата, то в тексте реферата она опускается, при этом последовательность изложения сохраняется.
%%
%% Оптимальный объем текста реферата — 850 печатных знаков, но не более одной страницы машинописного текста.
%% Реферат следует оформлять в соответствии с 6.12.
%%
%% 5.3.3 Примеры составления рефератов к отчету о НИР приведены в приложении В.


\StructuralElement{Реферат}

%% ГОСТ 7.32-2017
%% 5.4 Содержание
%%
%% 5.4.1 Содержание включает введение, наименование всех разделов и подразделов, пунктов (если они имеют наименование), заключение, список использованных источников и наименования приложений с указанием номеров страниц, с которых начинаются эти элементы отчёта о НИР.
%%
%% В элементе "СОДЕРЖАНИЕ" приводят наименования структурных элементов работы, порядковые номера и заголовки разделов, подразделов (при необходимости — пунктов) основной части работы, обозначения и заголовки её приложений (при наличии приложений).
%% После заголовка каждого элемента ставят отточие и приводят номер страницы работы, на которой начинается данный структурный элемент.
%%
%% Обозначения подразделов приводят после абзацного отступа, равного двум знакам, относительно обозначения разделов.
%% Обозначения пунктов приводят после абзацного отступа, равного четырём знакам относительно обозначения разделов.
%%
%% При необходимости продолжение записи заголовка раздела, подраздела или пункта на второй (последующей) строке выполняют, начиная от уровня начала этого заголовка на первой строке, а продолжение записи заголовка приложения — от уровня записи обозначения этого приложения.
%%
%% 5.4.2 При составлении отчёта, состоящего из двух и более книг, в каждой из них должно быть приведено своё содержание.
%% При этом в первой книге помещают содержание всего отчёта с указанием номеров книг, в последующих — только содержание соответствующей книги.
%% Допускается в первой книге вместо содержания последующих книг указывать только их наименования.
%%
%% 5.4.3 Для отчёта о НИР объёмом не более 10 страниц содержание допускается не составлять.
%%
%% 5.4.4 Содержание следует оформлять в соответствии с 6.13.


\tableofcontents

%%% ГОСТ 7.32-2017
%% 5.5 Термины и определения
%%
%% 5.5.1 Структурный элемент "ТЕРМИНЫ И ОПРЕДЕЛЕНИЯ" содержит определения, необходимые для уточнения или установления терминов, используемых в отчете о НИР.
%%
%% 5.5.2 Перечень терминов и определений начинают со слов: "В настоящем отчете о НИР применяют следующие термины с соответствующими определениями".
%%
%% 5.5.3 Термины и определения следует оформлять в соответствии с 6.14.


\StructuralElement{Термины и определения}

%%% ГОСТ 7.32-2017
%% 5.6 Перечень сокращений и обозначений
%%
%% 5.6.1 Структурный элемент "ПЕРЕЧЕНЬ СОКРАЩЕНИЙ И ОБОЗНАЧЕНИЙ" начинают со слов: "В настоящем отчете о НИР применяют следующие сокращения и обозначения".
%%
%% 5.6.2 Если в отчете используют более трех условных обозначений, требующих пояснения (включая специальные сокращения слов и словосочетаний, обозначения единиц физических величин и другие специальные символы), составляется их перечень, в котором для каждого обозначения приводят необходимые сведения.
%%
%% Допускается определения, обозначения и сокращения приводить в одном структурном элементе "ОПРЕДЕЛЕНИЯ, ОБОЗНАЧЕНИЯ И СОКРАЩЕНИЯ".
%%
%% 5.6.3 Если условных обозначений в отчете приведено менее трех, отдельный перечень не составляют, а необходимые сведения указывают в тексте отчета или в подстрочном примечании при первом упоминании.
%%
%% 5.6.4 Перечень сокращений и обозначений следует оформлять в соответствии с 6.15.


\StructuralElement{Перечень сокращений и обозначений}

%% ГОСТ 7.32-2017
%% 5.7 Введение
%%
%% 5.7.1 Введение должно содержать оценку современного состояния решаемой научно-технической проблемы, основание и исходные данные для разработки темы, обоснование необходимости проведения НИР, сведения о планируемом научно-техническом уровне разработки, о патентных исследованиях и выводы из них, сведения о метрологическом обеспечении НИР.
%% Во введении должны быть отражены актуальность и новизна темы, связь данной работы с другими научно-исследовательскими работами.
%%
%% 5.7.2 Во введении промежуточного отчета по этапу НИР должны быть указаны цели и задачи исследований, выполненных на данном этапе, их место в выполнении отчета о НИР в целом.
%%
%% 5.7.3 Во введении заключительного отчета о НИР приводят перечень наименований всех подготовленных промежуточных отчетов по этапам и их регистрационные номера, если они были представлены в соответствующий орган [1] для регистрации.
%%   [1] В Российской Федерации — ЦИТиС, который присваивает эти номера при представлении промежуточного отчета на регистрацию.


%% Методические указания к выполнению, оформлению и защите выпускной квалификационной работы бакалавра
%% 2.3 Введение
%%
%% Во введении обосновывается актуальность выбранной темы (со ссылками на монографии, научные статьи), формулируется цель работы («Целью работы является...») и перечисляются задачи, которые необходимо решить для достижения этой цели («Для достижения поставленной цели необходимо решить следующие задачи...»)
%%
%% Среди задач, как правило, выделяют аналитические, конструкторские, технологические и исследовательские.
%% Решение этих задач описывается в соответствующих разделах.
%%
%% Рекомендуемый объём введения 2—3 страницы.


\StructuralElement{Введение}

% https://moluch.ru/young/archive/15/1158/
В современном мире одну из лидирующих позиций занимает информационное пространство — это публичная площадка в сети Интернет, где человек излагает свои мысли.
Множество людей наблюдает за подобными публичными площадками и людьми, которые их ведут, в различных социальных сетях.

Один из наиболее интересных видов информационного пространства~— блог.
Термин «блог» произошёл от английского weblog («logging the web»~— записывать события в сеть).
Впервые его использовал американский программист Йори Баргер в 1997 году для обозначения сетевого дневника \cite{lagoshina}.

Блог — это личный дневник, который размещается в интернете, имеет яркую индивидуальность, оригинальное содержание и свою аудиторию.
В отличие от реальных дневников, которые читают только сами авторы, записи интернет-дневников принято выкладывать на всеобщее обозрение.

Людей, ведущих блог, называют блогерами.
Блогер — это любой человек, который ведёт собственный электронный дневник и является его администратором.

\textbf{Целью} данной курсовой работы является платформы для ведения онлайн-дневников (блогов).
В рамках выполнения проекта необходимо решить следующие задачи:
\begin{itemize}
	\item провести анализ существующих систем;
	\item формализовать задачу в виде определения необходимого функционала;
	\item спроектировать архитектуру системы и ее интерфейса.
	\item программно реализовать и протестировать спроектированную систему;
\end{itemize}


\begin{MainPart}
	%% Методические указания к выполнению, оформлению и защите выпускной квалификационной работы бакалавра
%% 2.4 Аналитический раздел
%%
%% В данном разделе расчётно-пояснительной записки проводится анализ предметной области и выделяется основной объект исследования.
%% Если формализовать предметную область с помощью математической модели не удаётся и при этом она сложна для понимания, то для отображения происходящих в ней процессов необходимо использовать методологию IDEF0, а для описания сущностей предметной области и взаимосвязей между ними — ER-модель.
%%
%% Затем выполняется обзор существующих методов и алгоритмов решения идентифицированной проблемы предметной области (опять же с обязательными ссылками на научные источники: монографии, статьи и др.) и их программных реализаций (при наличии), анализируются достоинства и недостатки каждого из них.
%% Выполненный обзор должен позволить объективно оценить актуальное состояние изучаемой проблемы.
%% Результаты проведённого анализа по возможности классифицируются и оформляются в табличной форме.
%%
%% На основе выполненного анализа обосновывается необходимость разработки нового или адаптации существующего метода или алгоритма.
%%
%% Если же целью анализа являлся отбор (на основе чётко сформулированных критериев) тех методов и алгоритмов, которые наиболее эффективно решают поставленную задачу, то форма представления результата должна подтвердить обоснованность сделанного выбора, в том числе — полноту и корректность предложенных автором критериев отбора.
%%
%% Одним из основных выводов аналитического раздела должно стать формализованное описание проблемы предметной области, на решение которой будет направлен данный проект, включающее в себя:
%% — описание входных и выходных данных;
%% — указание ограничений, в рамках которых будет разработан новый, адаптирован существующий или просто реализован метод или алгоритм;
%% — описание критериев сравнения нескольких реализаций метода или алгоритма;
%% — описание способов тестирования разработанного, адаптированного или реализованного метода или алгоритма;
%% — описание функциональных требований к разрабатываемому программному обеспечению,
%% при этом в зависимости от направления работы отдельные пункты могут отсутствовать.
%%
%% Если в результате работы будет создано программное обеспечение, реализующее большое количество типичных способов взаимодействия с пользователем, необходимо каждый из этих способов описать с помощью диаграммы прецедентов [4, 5].
%%
%% Рекомендуемый объём аналитического раздела 25—30 страниц.


\chapter{Аналитический раздел}

В данном разделе приводится краткий обзор предметной области.

\section{Существующие аналоги}

Среди аналогов разрабатываемой системы можно отметить «LiveJournal», «Лайфхакер» и «Blogger».

\subsection{LiveJournal}

«Живой Журнал», «ЖЖ» (англ. LiveJournal, LJ) \cite{LJ} — блог-платформа для ведения онлайн-дневников (блогов), а также отдельный персональный блог, размещённый на этой платформе.
Предоставляет возможность публиковать свои и комментировать чужие записи, вести коллективные блоги («сообщества»), добавлять в друзья («френдить») других пользователей и следить за их записями в «ленте друзей» («френдленте»).
До декабря 2012 года отличался отсутствием обязательной рекламы в бесплатных блогах.

До конца декабря 2016 года «Живой Журнал» подчинялся американским законам, так как его серверы находились в США и система принадлежит американской компании LiveJournal, Inc.

\subsection{Лайфхакер}

Лайфхакер \cite{lifehacker} — это ежедневно обновляемый англоязычный интернет-блог, который специализируется на новостях в области программного и аппаратного обеспечение для Microsoft Windows, Mac OS и Linux, основанный в 2005 году.
Веб-портал является частью Kotaku.com. Победитель конкурса блог года 2007 в номинации «Лучший групповой блог года», а также вошёл в список лучших блогов года 2009 по версии журнала Time.

По статистике Alexa.com на 25 сентября 2010 года, Лайфхакер находится на 617 месте по посещаемости в мире, 256 в США, 316 в Канаде, 351 в Южной Африке и 3342 в России.

\subsection{Blogger}

Blogger \cite{blogger} — веб-сервис для ведения блогов, с помощью которого любой пользователь может завести свой блог, не прибегая к программированию и не заботясь об установке и настройке программного обеспечения. Blogger был создан компанией Pyra Labs, которой сейчас владеет Google.

До 1 мая 2010 посты Blogger могли автоматически переноситься на хостинг владельца блога при помощи FTP или SFTP.

\subsection{Основные требования к системе}

К недостаткам рассмотренных аналогов можно отнести то, что данные порталы являются зарубежными, что при текущих обстоятельствах ставит под вопрос стабильность работы на территории РФ.

Отличием данной разработки является сфокусированность на рускоязычную аудиторию.
К преимуществам можно отнести стабильность работы на территории России.

Главное назначение разрабатываемой системы — предоставление пользователю возможности ведения собственного онлайн-дневника; просмотра публикаций других пользователей; формирования личного круга интересов посредством подписок на определённых авторов и теги; обсуждения публикаций в комментариях к ним; просмотра истории прочитанных публикаций.
У пользователей должна быть возможность оценивать публикации и комментарии, сортировать их по дате и рейтингу.
Система должна позволять фильтровать публикации по определённым тегам, а также формировать топ-лист самых популярных публикаций за всё время.

Разрабатываемая система должна:
\begin{enumerate}
	\item обеспечивать
	\begin{enumerate}
		\item регистрацию и авторизацию пользователей с валидацией вводимых данных через интерфейс приложения;
		\item аутентификацию пользователей;
		\item разделение пользователей на три роли:
		\begin{itemize}
			\item Гость (неавторизованный Пользователь),
			\item Пользователь,
			\item Администратор;
		\end{itemize}
		причём Пользователю доступны все функции Гостя, Администратору — все функции Пользователя;
	\end{enumerate}
	\item предоставлять Гостю следующие функции:
	\begin{enumerate}
		\item получение списка публикаций
		\begin{itemize}
			\item по автору,
			\item по тегу,
			\item всех авторов;
		\end{itemize}
		\item просмотр полного текста публикаций и комментариев к ней;
		\item сортировка публикаций/комментариев по дате/рейтингу;
	\end{enumerate}
	\item предоставлять Пользователю следующие функции:
	\begin{enumerate}
		\item добавление публикаций в свой онлайн-дневник;
		\item редактирование публикаций из своего онлайн-дневника;
		\item удаление публикаций из своего онлайн-дневника;
		\item управление подписками на других авторов;
		\item управление подписками на теги;
		\item получение списка публикаций по авторам и тегам, на которые подписан Пользователь;
		\item добавление комментария к публикации;
		\item редактирование своего комментария;
		\item удаление своего комментария;
		\item оценивать «плюсом» или «минусом» публикации/комментарии;
	\end{enumerate}
	\item предоставлять Администратору следующие функции:
	\begin{itemize}
		\item отображение статистики просмотров публикаций.
	\end{itemize}
\end{enumerate}

Графически сценарии функционирования системы можно представить при помощи диаграмм прецедентов.
Они позволяют схематично отобразить типичные сценарии взаимодействия между клиентами и приложением.
В системе выделены 3 основных роли: Гость, Пользователь и Администратор, диаграммы прецедентов для этих ролей изображены на рисунках \ref{img:use-case-diagram-guest}, \ref{img:use-case-diagram-user2} и \ref{img:use-case-diagram-admin2}.

\imgH{scale=1.062}{use-case-diagram-guest}{Диаграмма прецедентов с точки зрения Гостя}

\imgH{scale=1.062}{use-case-diagram-user2}{Диаграмма прецедентов с точки зрения Пользователя}

\imgH{scale=1.062}{use-case-diagram-admin2}{Диаграмма прецедентов с точки зрения Администратора}

%\section{Выводы}

	%% Методические указания к выполнению, оформлению и защите выпускной квалификационной работы бакалавра
%% 2.5 Конструкторский раздел
%%
%% В конструкторском разделе описывается разрабатываемый и/или модифицируемый метод или алгоритм.
%%
%% В случае если в бакалаврском проекте разрабатывается новый метод или алгоритм, необходимо подробно изложить их суть, привести всё необходимые для их реализации математические выкладки, обосновать последовательность этапов выполнения.
%% При этом для каждого этапа следует выделить необходимые исходные данные и получаемые результаты.
%%
%% При использовании известного алгоритма следует указать специфические особенности его практической реализации, присущие решаемой задаче, и пути их решения в ходе программирования.
%% Для описания метода или алгоритма необходимо выбрать наиболее подходящую форму записи (схема (ГОСТ 19.701-90), диаграмма деятельности, псевдокод и т. п.).
%% Учитывая, что на эффективность алгоритма непосредственно влияют используемые структуры данных, в данном разделе РПЗ целесообразно провести сравнительный анализ структур, которые могут быть применены в рамках программной реализации выбранного алгоритма, и обосновать выбор одной из них.
%% В конце описания разработанного и/или модифицируемого алгоритма должны быть приведены выбранные способы тестирования и сами тесты.
%%
%% Перед формированием тестовых наборов данных целесообразно указать выделенные классы эквивалентности.
%% В данной части расчётно-пояснительной записки могут также выполняться расчёты для определения объёмов памяти, необходимой для хранения данных, промежуточных и окончательных результатов работы программы, а также расчёты, позволяющие оценить время решения задачи на ЭВМ.
%% Эти результаты могут использоваться для обоснования правильности выбора метода и/или алгоритма из имеющихся альтернативных вариантов, а также для оценки возможности практически реализовать поставленную задачу на имеющейся технической базе.
%%
%% Другой важный момент, который должен найти своё отражение в конструкторском разделе, это описание структуры разрабатываемого программного обеспечения.
%% Обычно оно включает в себя:
%% — описание общей структуры — определение основных частей (компонентов) и их взаимосвязей по управлению и по данным;
%% — декомпозицию компонентов и построение структурных иерархий;
%% — проектирование компонентов.
%%
%% Для графического представления такого описания, если есть необходимость, следует использовать:
%% — функциональную модель IDEF0 с декомпозицией решения исходной задачи на несколько уровней (разрабатываемые модули обычно играют роль механизмов);
%% — спецификации компонентов (процессов);
%% — модель данных (ER-диаграмма);
%% — диаграмму классов;
%% — диаграмму компонентов;
%% — диаграмму переходов состояний (конечный автомат), характеризующих поведение системы во времени.
%%
%% Рекомендуемый объем конструкторского раздела 25—30 страниц.


\chapter{Конструкторский раздел}

\section{Архитектура системы}

В данной работе используется микросервисная архитектура, позволяющая разделить программную систему на небольшие, слабо связанные и легко изменяемые модули.

Было выделено 6 микросервисов.
\begin{itemize}
	\item Сервис пользовательского интерфейса. Данный сервис принимает запросы от пользователей и в зависимости от них выполняет запросы сервису-координатору и сервису авторизации для получения токен.
	\item Сервис-координатор должен реализовать диспетчеризацию запросов.
	\item Сервис публикаций — отвечает за добавление публикаций и хранение информации о них.
	\item Сервис подписок — осуществляет формирование персонального информационного окружения пользователя.
	\item Сервис регистрации и авторизации.
	\item Сервис статистики.
\end{itemize}

Каждый из микросервисов кроме сервиса пользовательского интерфейса и сервиса-координатора имеет свою базу данных и не обращается к базам данных других сервисов.

Топология разрабатываемой системы изображена на рисунке \ref{img:topology}.

\imgH{width=\linewidth}{topology}{Топология системы}

\section{Сущности системы}

На основе функциональных требований к выделенным подсистемам, а также объектов, о которых необходимо хранить данные в системе, была разработана схема данных приложения.
Результат её проектирования отображён на условной ER-диаграмме, представленной на рисунке \ref{img:erd-chen}.

\imght{width=\linewidth}{erd-chen}{ER-диаграмма данных системы}

На следующей стадии проектирования, добавив в схему данных атрибуты сущностей, получаем схему базы данных, которая изображена на рисунке \ref{img:erd-crow's-foot2}.

\imgH{width=\linewidth}{erd-crow's-foot2}{Схема базы данных системы}

\section{Взаимодействие систем}

Для описания поведения компонентов системы на единой оси времени используются диаграммы последовательности действий, при помощи которых можно описать последовательность действий для каждого прецедента, необходимую для
достижения цели.
Например, на рисунке \ref{img:sequence-diagram} изображён процесс получения списка публикаций пользователем на основе его подписок.

\imgH{width=\linewidth}{sequence-diagram}{Диаграмма последовательности действий при запросе пользователем публикаций на основе его подписок}

Сервис-координатор отправляет запрос сервису подписок на получение списка авторов и тегов, на которые подписан пользователь.
После получения данных сервис-координатор обращается к сервису публикаций, чтобы получить список публикаций по переданным авторам и тегам.
После окончания данной процедуры главный сервис ранжирует список по времени публикации, формирует веб-страницу и возвращает её пользователю.


%\section{Выводы}

	%% Методические указания к выполнению, оформлению и защите выпускной квалификационной работы бакалавра
%% 2.6 Технологический раздел
%%
%% Технологический раздел содержит обоснованный выбор средств программной реализации, описание основных (нетривиальных) моментов разработки и методики тестирования созданного программного обеспечения.
%%
%% В этом же разделе описывается информация, необходимая для сборки и запуска разработанного программного обеспечения, форматы входных, выходных и конфигурационных файлов (если такие имеются), а также интерфейс пользователя и руководство пользователя.
%%
%% Если для правильного функционирования разработанного программного обеспечения требуется некоторая инфраструктура (веб-приложение, база данных, серверное приложение), уместно представить её с помощью диаграммы развёртывания UML.
%%
%% Как уже говорилось, часть технологического раздела должна быть посвящена тестированию разработанного программного обеспечения.
%%
%% Модульное тестирование описывается в технологическом разделе.
%%
%% Системное тестирование может быть описано в технологическом или экспериментальном разделах, в зависимости от глубины его реализации и тематики бакалаврской работы.
%%
%% При проведении тестирования разработанного программного обеспечения следует широко использовать специализированные программные приложения: различные статические анализаторы кода (например, clang); для тестирования утечек памяти в языках программирования, где отсутствует автоматическая «сборка мусора», Valgrind, Doctor Memory и их аналоги, и т. п.
%%
%% Рекомендуемый объём технологического раздела 20—25 страниц.


\chapter{Технологический раздел}

\section{Структура баз данных}

На основе описанных выше сущностей для каждого из сервисов были разработаны спецификации таблиц баз данных, приведённых на рисунке \ref{img:erd-crow's-foot}.

\begin{table}[H]
	\caption{Спецификация таблицы User}
	\begin{tabular}{|l|l|l|}
		\hline
		\multicolumn{1}{|c|}{\textbf{Имя атрибута}} & \multicolumn{1}{c|}{\textbf{Тип атрибута}} & \multicolumn{1}{c|}{\textbf{Описание атрибута}} \\ \hline
		id                                          & UUID4                                      & Идентификатор пользователя                      \\ \hline
		email                                       & string                                     & Электронная почта                               \\ \hline
		password\_hash                              & string                                     & Хеш пароля                                      \\ \hline
		full\_name                                  & string                                     & ФИО                                             \\ \hline
		role                                        & string                                     & Роль \\ \hline
	\end{tabular}
\end{table}

\begin{table}[H]
	\caption{Спецификация таблицы Tag}
	\begin{tabular}{|l|l|l|}
		\hline
		\multicolumn{1}{|c|}{\textbf{Имя атрибута}} & \multicolumn{1}{c|}{\textbf{Тип атрибута}} & \multicolumn{1}{c|}{\textbf{Описание атрибута}} \\ \hline
		id                                          & UUID4                                      & Идентификатор тега                              \\ \hline
		name                                        & string                                     & Категория                                       \\ \hline
	\end{tabular}
\end{table}

\begin{table}[H]
	\caption{Спецификация таблицы Publication}
	\begin{tabular}{|l|l|l|}
		\hline
		\multicolumn{1}{|c|}{\textbf{Имя атрибута}} & \multicolumn{1}{c|}{\textbf{Тип атрибута}} & \multicolumn{1}{c|}{\textbf{Описание атрибута}} \\ \hline
		id                                          & UUID4                                      & Идентификатор публикации                        \\ \hline
		author\_id                                  & UUID4                                      & Идентификатор автора                            \\ \hline
		title                                       & string                                     & Заголовок                                       \\ \hline
		body                                        & string                                     & Тело                                            \\ \hline
		pub\_date                                   & timestamp                                  & Дата публикации                                 \\ \hline
		rating                                      & int                                        & Рейтинг                                         \\ \hline
	\end{tabular}
\end{table}

\begin{table}[H]
	\caption{Спецификация таблицы Comment}
	\begin{tabular}{|l|l|l|}
		\hline
		\multicolumn{1}{|c|}{\textbf{Имя атрибута}} & \multicolumn{1}{c|}{\textbf{Тип атрибута}} & \multicolumn{1}{c|}{\textbf{Описание атрибута}} \\ \hline
		id                                          & UUID4                                      & Идентификатор комментария                       \\ \hline
		author\_id                                  & UUID4                                      & Идентификатор автора                            \\ \hline
		publication\_id                             & UUID4                                      & Идентификатор публикации                        \\ \hline
		body                                        & string                                     & Тело                                            \\ \hline
		pub\_date                                   & timestamp                                  & Дата публикации комментария                     \\ \hline
		rating                                      & int                                        & Рейтинг                                         \\ \hline
	\end{tabular}
\end{table}

\begin{table}[H]
	\caption{Спецификация таблицы Subscription}
	\begin{tabular}{|l|l|l|}
		\hline
		\multicolumn{1}{|c|}{\textbf{Имя атрибута}} & \multicolumn{1}{c|}{\textbf{Тип атрибута}} & \multicolumn{1}{c|}{\textbf{Описание атрибута}} \\ \hline
		id                                          & UUID4                                      & Идентификатор сущности                          \\ \hline
		user\_id                                    & UUID4                                      & Идентификатор пользователя                      \\ \hline
		sub\_id                                     & UUID4                                      & Идентификатор объекта подписки                  \\ \hline
		type                                        & string                                     & Тип подписки (Тег или Автор)                    \\ \hline
	\end{tabular}
\end{table}

\begin{table}[H]
	\caption{Спецификация таблицы Publication\_tag}
	\begin{tabular}{|l|l|l|}
		\hline
		\multicolumn{1}{|c|}{\textbf{Имя атрибута}} & \multicolumn{1}{c|}{\textbf{Тип атрибута}} & \multicolumn{1}{c|}{\textbf{Описание атрибута}} \\ \hline
		id                                          & UUID4                                      & Идентификатор сущности                          \\ \hline
		publication\_id                             & UUID4                                      & Идентификатор публикации                        \\ \hline
		tag\_id                                     & UUID4                                      & Идентификатор тега                              \\ \hline
	\end{tabular}
\end{table}

\section{Сборка и деплой системы}

Для автоматической сборки проекта используется функциональность Github Actions \cite{github-actions}.
При отправке изменений в master ветку репозитория автоматически запускается сборка и тестирование каждого из микросервисов.
Конфигурация запуска автоматической сборки и тестирования
представлена на листинге \ref{lst:ci-cd}.

\begin{lstlisting}[
	caption={Конфигурация запуска автоматической сборки и тестирования},
	label={lst:ci-cd},
]
name: Django

on:
  push:
    branches: [master]
  pull_request:
    branches: [master]

jobs:
  build:
    runs-on: ubuntu-latest
    steps:
    - uses: actions/checkout@v3

    - name: Set up python
      uses: actions/setup-python@v4
      with:
        python-version: '3.10.4'

    - name: Install requirements
      run: python -m pip install -r requirements.txt

    - name: Migrate Session Service
      run: python backend_session/manage.py migrate

    - name: Migrate Publication Service
      run: python backend_publication/manage.py migrate

    - name: Migrate Subscription Service
      run: python backend_subscription/manage.py migrate

    - name: Migrate Statistics Service
      run: python backend_statistics/manage.py migrate

    - name: Test Gateway Service
      run: cd backend_gateway && python manage.py test

    - name: Test Session Service
      run: cd backend_session && python manage.py test

    - name: Test Publication Service
      run: cd backend_publication && python manage.py test

    - name: Test Subscription Service
      run: cd backend_subscription && python manage.py test

    - name: Test Statistics Service
      run: cd backend_statistics && python manage.py test
\end{lstlisting}

Для развертывания системы используется docker compose, каждый сервис и база данных каждого сервиса разворачиваются в отдельном контейнере.
Все контейнеры микросервисы связаны общей сетью, каждый микросервис связан отдельной сетью с контейнером с его базой данных.

\section{Деградация системы}

При отказе сервиса статистики должны продолжить полноценно работать все пользовательские сценарии за исключением просмотра статистики.

При отказе очередей для сбора статистики должны продолжить полноценно работать все пользовательские сценарии, одна статистика в это время собираться не будет.

При отказе сервиса подписок должны продолжить работать следующие пользовательские сценарии, кроме следующих:

\begin{itemize}
	\item подписка на автора;
	\item подписка на тэг;
	\item отписка от автора;
	\item отписка от тега;
	\item просмотр своей ленты публикаций по подпискам.
\end{itemize}


%\section{Выводы}

	%%% Методические указания к выполнению, оформлению и защите выпускной квалификационной работы бакалавра
%% 2.7 Экспериментальный раздел
%%
%% Данный раздел содержит описание проведенных экспериментов и их результаты.
%% Должно быть обязательно указано, какую цель ставил перед собой автор работы при планировании экспериментов, какие предположения/гипотезы он надеялся подтвердить и/или опровергнуть с их помощью.
%% Результаты оформляются в виде графиков, диаграмм и/или таблиц.
%%
%% Здесь же может быть проведено качественное и количественное сравнение с аналогами.
%%
%% Рекомендуемый объем экспериментального раздела 10—15 страниц.


\chapter{Исследовательский раздел}

\section{Выводы}

\end{MainPart}


%% ГОСТ 7.32-2017
%% 5.9 Заключение
%%
%% Заключение должно содержать:
%% - краткие выводы по результатам выполненной НИР или отдельных ее этапов;
%% - оценку полноты решений поставленных задач;
%% - разработку рекомендаций и исходных данных по конкретному использованию результатов НИР;
%% - результаты оценки технико-экономической эффективности внедрения;
%% - результаты оценки научно-технического уровня выполненной НИР в сравнении с лучшими достижениями в этой области.

%% Методические указания к выполнению, оформлению и защите выпускной квалификационной работы бакалавра
%% 2.10 Заключение
%%
%% Заключение содержит краткие выводы по всей работе и оценку полноты решения поставленной задачи.


\StructuralElement{Заключение}

В ходе работы над данным проектом был проведен обзор существующих систем и формализованы требования к проекту, спроектирована архитектура
системы и её интерфейс.

В результате чего была реализована реализована система
ведения онлайн-дневников (блогов).

%% ГОСТ 7.32-2017
%% 5.10 Список использованных источников
%%
%% 5.10.1 Список должен содержать сведения об источниках, использованных при составлении отчета.
%% Сведения об источниках приводятся в соответствии с требованиями ГОСТ 7.1, ГОСТ 7.80, ГОСТ 7.82.
%%
%% 5.10.2 Список использованных источников должен включать библиографические записи на документы, использованные при составлении отчета, ссылки на которые оформляют арабскими цифрами в квадратных скобках.
%% Список использованных источников оформляют в соответствии с 6.16.


\addcontentsline{toc}{chapter}{СПИСОК ИСПОЛЬЗОВАННЫХ ИСТОЧНИКОВ}

\renewcommand\bibname{Список использованных источников}

\begin{thebibliography}{3}
	\bibitem{lagoshina} Лагошина, М. С. Роль блогов и блогеров в сети Интернет / М. С. Лагошина, Ю. А. Саева. — Текст: непосредственный // Юный ученый. — 2018. — № 1.1 (15.1). — С. 52-53.
	\bibitem{LJ} LiveJournal. [Электронный ресурс]. URL: \url{https://www.livejournal.com/} (дата обращения: 10.04.2022)
	\bibitem{lifehacker} Лайфхакер. [Электронный ресурс]. URL: \url{https://lifehacker.com/} (дата обращения: 10.04.2022)
	\bibitem{blogger} Blogger. [Электронный ресурс]. URL: \url{https://blogger.com/} (дата обращения: 10.04.2022)
	\bibitem{github-actions} Github Actions [Электронный ресурс]. URL:
	\url{https://docs.github.com/en/actions} (дата обращения: 10.04.2022)
\end{thebibliography}

%%% ГОСТ 7.32-2017
%% 5.11 Приложения
%%
%% 5.11.1 В приложения рекомендуется включать материалы, дополняющие текст отчета, связанные с выполненной НИР, если они не могут быть включены в основную часть.
%%
%% В приложения могут быть включены:
%% - дополнительные материалы к отчету;
%% - промежуточные математические доказательства и расчеты;
%% - таблицы вспомогательных цифровых данных;
%% - протоколы испытаний;
%% - заключение метрологической экспертизы;
%% - инструкции, методики, описания алгоритмов и программ, разработанных в процессе выполнения НИР;
%% - иллюстрации вспомогательного характера;
%% - копии технического задания на НИР, программы работ или другие исходные документы для выполнения НИР;
%% - протокол рассмотрения результатов выполненной НИР на научно-техническом совете;
%% - акты внедрения результатов НИР или их копии;
%% - копии охранных документов.
%%
%% 5.11.2 Приложения к отчету о НИР, в составе которых предусмотрено проведение патентных исследований, могут быть включены в отчет о патентных исследованиях, оформленный по ГОСТ 15.011, библиографический список публикаций и патентных документов, полученных в результате выполнения НИР, который должен быть оформлен по ГОСТ 7.1, ГОСТ 7.80, ГОСТ 7.82.
%%
%% 5.11.3 Приложения оформляются в соответствии с 6.17.

%% Методические указания к выполнению, оформлению и защите выпускной квалификационной работы бакалавра
%% 2.12 Приложения
%%
%% Приложения состоят из вспомогательного материала, на который в основной части бакалаврской работы имеются ссылки.
%% Приложением оформляют различные схемы, листинг программ, наборы тестов и др.
%%
%% В тексте РПЗ на все приложения должны быть даны ссылки.


\StructuralElement{Приложение А}


\end{document}
