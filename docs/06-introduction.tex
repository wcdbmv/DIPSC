%% ГОСТ 7.32-2017
%% 5.7 Введение
%%
%% 5.7.1 Введение должно содержать оценку современного состояния решаемой научно-технической проблемы, основание и исходные данные для разработки темы, обоснование необходимости проведения НИР, сведения о планируемом научно-техническом уровне разработки, о патентных исследованиях и выводы из них, сведения о метрологическом обеспечении НИР.
%% Во введении должны быть отражены актуальность и новизна темы, связь данной работы с другими научно-исследовательскими работами.
%%
%% 5.7.2 Во введении промежуточного отчета по этапу НИР должны быть указаны цели и задачи исследований, выполненных на данном этапе, их место в выполнении отчета о НИР в целом.
%%
%% 5.7.3 Во введении заключительного отчета о НИР приводят перечень наименований всех подготовленных промежуточных отчетов по этапам и их регистрационные номера, если они были представлены в соответствующий орган [1] для регистрации.
%%   [1] В Российской Федерации — ЦИТиС, который присваивает эти номера при представлении промежуточного отчета на регистрацию.


%% Методические указания к выполнению, оформлению и защите выпускной квалификационной работы бакалавра
%% 2.3 Введение
%%
%% Во введении обосновывается актуальность выбранной темы (со ссылками на монографии, научные статьи), формулируется цель работы («Целью работы является...») и перечисляются задачи, которые необходимо решить для достижения этой цели («Для достижения поставленной цели необходимо решить следующие задачи...»)
%%
%% Среди задач, как правило, выделяют аналитические, конструкторские, технологические и исследовательские.
%% Решение этих задач описывается в соответствующих разделах.
%%
%% Рекомендуемый объём введения 2—3 страницы.


\StructuralElement{Введение}

% https://moluch.ru/young/archive/15/1158/
В современном мире одну из лидирующих позиций занимает информационное пространство — это публичная площадка в сети Интернет, где человек излагает свои мысли.
Множество людей наблюдает за подобными публичными площадками и людьми, которые их ведут, в различных социальных сетях.

Один из наиболее интересных видов информационного пространства~— блог.
Термин «блог» произошёл от английского weblog («logging the web»~— записывать события в сеть).
Впервые его использовал американский программист Йори Баргер в 1997 году для обозначения сетевого дневника \cite{lagoshina}.

Блог — это личный дневник, который размещается в интернете, имеет яркую индивидуальность, оригинальное содержание и свою аудиторию.
В отличие от реальных дневников, которые читают только сами авторы, записи интернет-дневников принято выкладывать на всеобщее обозрение.

Людей, ведущих блог, называют блогерами.
Блогер — это любой человек, который ведёт собственный электронный дневник и является его администратором.

\textbf{Целью} данной курсовой работы является платформы для ведения онлайн-дневников (блогов).
В рамках выполнения проекта необходимо решить следующие задачи:
\begin{itemize}
	\item провести анализ существующих систем;
	\item формализовать задачу в виде определения необходимого функционала;
	\item спроектировать архитектуру системы и ее интерфейса.
	\item программно реализовать и протестировать спроектированную систему;
\end{itemize}
