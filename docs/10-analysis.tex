%% Методические указания к выполнению, оформлению и защите выпускной квалификационной работы бакалавра
%% 2.4 Аналитический раздел
%%
%% В данном разделе расчётно-пояснительной записки проводится анализ предметной области и выделяется основной объект исследования.
%% Если формализовать предметную область с помощью математической модели не удаётся и при этом она сложна для понимания, то для отображения происходящих в ней процессов необходимо использовать методологию IDEF0, а для описания сущностей предметной области и взаимосвязей между ними — ER-модель.
%%
%% Затем выполняется обзор существующих методов и алгоритмов решения идентифицированной проблемы предметной области (опять же с обязательными ссылками на научные источники: монографии, статьи и др.) и их программных реализаций (при наличии), анализируются достоинства и недостатки каждого из них.
%% Выполненный обзор должен позволить объективно оценить актуальное состояние изучаемой проблемы.
%% Результаты проведённого анализа по возможности классифицируются и оформляются в табличной форме.
%%
%% На основе выполненного анализа обосновывается необходимость разработки нового или адаптации существующего метода или алгоритма.
%%
%% Если же целью анализа являлся отбор (на основе чётко сформулированных критериев) тех методов и алгоритмов, которые наиболее эффективно решают поставленную задачу, то форма представления результата должна подтвердить обоснованность сделанного выбора, в том числе — полноту и корректность предложенных автором критериев отбора.
%%
%% Одним из основных выводов аналитического раздела должно стать формализованное описание проблемы предметной области, на решение которой будет направлен данный проект, включающее в себя:
%% — описание входных и выходных данных;
%% — указание ограничений, в рамках которых будет разработан новый, адаптирован существующий или просто реализован метод или алгоритм;
%% — описание критериев сравнения нескольких реализаций метода или алгоритма;
%% — описание способов тестирования разработанного, адаптированного или реализованного метода или алгоритма;
%% — описание функциональных требований к разрабатываемому программному обеспечению,
%% при этом в зависимости от направления работы отдельные пункты могут отсутствовать.
%%
%% Если в результате работы будет создано программное обеспечение, реализующее большое количество типичных способов взаимодействия с пользователем, необходимо каждый из этих способов описать с помощью диаграммы прецедентов [4, 5].
%%
%% Рекомендуемый объём аналитического раздела 25—30 страниц.


\chapter{Аналитический раздел}

В данном разделе приводится краткий обзор предметной области.

\section{Существующие аналоги}

Среди аналогов разрабатываемой системы можно отметить «LiveJournal», «Лайфхакер» и «Blogger».

\subsection{LiveJournal}

«Живой Журнал», «ЖЖ» (англ. LiveJournal, LJ) \cite{LJ} — блог-платформа для ведения онлайн-дневников (блогов), а также отдельный персональный блог, размещённый на этой платформе.
Предоставляет возможность публиковать свои и комментировать чужие записи, вести коллективные блоги («сообщества»), добавлять в друзья («френдить») других пользователей и следить за их записями в «ленте друзей» («френдленте»).
До декабря 2012 года отличался отсутствием обязательной рекламы в бесплатных блогах.

До конца декабря 2016 года «Живой Журнал» подчинялся американским законам, так как его серверы находились в США и система принадлежит американской компании LiveJournal, Inc.

\subsection{Лайфхакер}

Лайфхакер \cite{lifehacker} — это ежедневно обновляемый англоязычный интернет-блог, который специализируется на новостях в области программного и аппаратного обеспечение для Microsoft Windows, Mac OS и Linux, основанный в 2005 году.
Веб-портал является частью Kotaku.com. Победитель конкурса блог года 2007 в номинации «Лучший групповой блог года», а также вошёл в список лучших блогов года 2009 по версии журнала Time.

По статистике Alexa.com на 25 сентября 2010 года, Лайфхакер находится на 617 месте по посещаемости в мире, 256 в США, 316 в Канаде, 351 в Южной Африке и 3342 в России.

\subsection{Blogger}

Blogger \cite{blogger} — веб-сервис для ведения блогов, с помощью которого любой пользователь может завести свой блог, не прибегая к программированию и не заботясь об установке и настройке программного обеспечения. Blogger был создан компанией Pyra Labs, которой сейчас владеет Google.

До 1 мая 2010 посты Blogger могли автоматически переноситься на хостинг владельца блога при помощи FTP или SFTP.

\subsection{Основные требования к системе}

К недостаткам рассмотренных аналогов можно отнести то, что данные порталы являются зарубежными, что при текущих обстоятельствах ставит под вопрос стабильность работы на территории РФ.

Отличием данной разработки является сфокусированность на рускоязычную аудиторию.
К преимуществам можно отнести стабильность работы на территории России.

Главное назначение разрабатываемой системы — предоставление пользователю возможности ведения собственного онлайн-дневника; просмотра публикаций других пользователей; формирования личного круга интересов посредством подписок на определённых авторов и теги; обсуждения публикаций в комментариях к ним; просмотра истории прочитанных публикаций.
У пользователей должна быть возможность оценивать публикации и комментарии, сортировать их по дате и рейтингу.
Система должна позволять фильтровать публикации по определённым тегам, а также формировать топ-лист самых популярных публикаций за всё время.

Разрабатываемая система должна:
\begin{enumerate}
	\item обеспечивать
	\begin{enumerate}
		\item регистрацию и авторизацию пользователей с валидацией вводимых данных через интерфейс приложения;
		\item аутентификацию пользователей;
		\item разделение пользователей на три роли:
		\begin{itemize}
			\item Гость (неавторизованный Пользователь),
			\item Пользователь,
			\item Администратор;
		\end{itemize}
		причём Пользователю доступны все функции Гостя, Администратору — все функции Пользователя;
	\end{enumerate}
	\item предоставлять Гостю следующие функции:
	\begin{enumerate}
		\item получение списка публикаций
		\begin{itemize}
			\item по автору,
			\item по тегу,
			\item всех авторов;
		\end{itemize}
		\item просмотр полного текста публикаций и комментариев к ней;
		\item сортировка публикаций/комментариев по дате/рейтингу;
	\end{enumerate}
	\item предоставлять Пользователю следующие функции:
	\begin{enumerate}
		\item добавление публикаций в свой онлайн-дневник;
		\item редактирование публикаций из своего онлайн-дневника;
		\item удаление публикаций из своего онлайн-дневника;
		\item управление подписками на других авторов;
		\item управление подписками на теги;
		\item получение списка публикаций по авторам и тегам, на которые подписан Пользователь;
		\item добавление комментария к публикации;
		\item редактирование своего комментария;
		\item удаление своего комментария;
		\item оценивать «плюсом» или «минусом» публикации/комментарии;
	\end{enumerate}
	\item предоставлять Администратору следующие функции:
	\begin{itemize}
		\item отображение статистики просмотров публикаций.
	\end{itemize}
\end{enumerate}

Графически сценарии функционирования системы можно представить при помощи диаграмм прецедентов.
Они позволяют схематично отобразить типичные сценарии взаимодействия между клиентами и приложением.
В системе выделены 3 основных роли: Гость, Пользователь и Администратор, диаграммы прецедентов для этих ролей изображены на рисунках \ref{img:use-case-diagram-guest}, \ref{img:use-case-diagram-user2} и \ref{img:use-case-diagram-admin2}.

\imgH{scale=1.062}{use-case-diagram-guest}{Диаграмма прецедентов с точки зрения Гостя}

\imgH{scale=1.062}{use-case-diagram-user2}{Диаграмма прецедентов с точки зрения Пользователя}

\imgH{scale=1.062}{use-case-diagram-admin2}{Диаграмма прецедентов с точки зрения Администратора}

%\section{Выводы}
