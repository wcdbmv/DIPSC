\documentclass{bmstu-gost-7-32}


\renewcommand*\thesection{\arabic{section}}


\begin{document}

\chapter*{Техническое задание}

%% Т. И. Вишневская, Т. Н. Романова
%% Практикум по разработке распределённых систем обработки информации
%% Данный раздел перечислен в 3.2.1
\section*{Глоссарий}

Латентность географического положения — увеличение времени отклика приложения, обуславливаемое географическим положением элементов системы или пользователя.

% https://ctrlok.com/post/web-server-percentiles/
Медиана времени отклика — максимальное время предоставления данных пользователю для каждого из как минимум половины пользователей.

Публикация — документ, доступный для массового использования (ГОСТ~7.0-99).

СОА — Сервис-ориентированная архитектура — модульный подход к разработке ПО, базирующийся на обеспечении удалённого по стандартизированным протоколам использования распределённых, слабо связанных легко заменяемых компонентов (сервисов) со стандартизированными интерфейсами.

Тег — идентификатор для категоризации, описания или поиска.

Узел системы — сервер, соединённый с другими устройствами как часть компьютерной сети.

UUID4 — формат представления уникальных идентификаторов в СОА.


\section{Введение}

%% ГОСТ 19.201-78
%% 2.1. В разделе "Введение" указывают наименование, краткую характеристику области применения программы или программного изделия и объекта, в котором используют программу или программное изделие.

%% Т. И. Вишневская, Т. Н. Романова
%% Лекции по технологиям программирования 2011
%% 1. Введение.
%% Здесь следует писать краткую характеристику предметной области (медицина, торговля, промышленность).
%% Можно даже нарисовать в виде картинки.
%% Есть торговое предприятие.
%% Есть поставщики и заказчики.
%% Тогда на введении, прямо на странице рисуют схематично всё это. Условно обозначаю, кто соучаствует в процессе.
%%
%% И, конечно, какая проблема стоит, какие решения есть (существующие разработки), какие у них недостатки.
%% Напр., устарела и т.п.
%% Т.е. почему возникла необходимость создания нового ПО, в самом общем виде — в том виде, в каком описывает заказчик.
%% Например «большая доля ручного труда, следует автоматизировать, чтобы снизить человеческий фактор ошибок и повысить производительность»

\subsection{Наименование программы}

Наименование программы — «Распределённая система ведения онлайн-дневников (блогов)» (далее — Система).

\subsection{Краткое описание предметной области}

% https://moluch.ru/young/archive/15/1158/
В современном мире одну из лидирующих позиций занимает информационное пространство — это публичная площадка в сети Интернет, где человек излагает свои мысли.
Множество людей наблюдает за подобными публичными площадками и людьми, которые их ведут, в различных социальных сетях.

Один из наиболее интересных видов информационного пространства — блог.
Термин «блог» произошёл от английского weblog («logging the web» — записывать события в сеть).
Впервые его использовал американский программист Йори Баргер в 1997 году для обозначения сетевого дневника.

Блог — это личный дневник, который размещается в интернете, имеет яркую индивидуальность, оригинальное содержание и свою аудиторию.
В отличие от реальных дневников, которые читают только сами авторы, записи интернет-дневников принято выкладывать на всеобщее обозрение.

Людей, ведущих блог, называют блогерами.
Блогер — это любой человек, который ведёт собственный электронный дневник и является его администратором.

%% Т. И. Вишневская, Т. Н. Романова
%% Практикум по разработке распределённых систем обработки информации
%% Данный раздел в перечислен в 3.2.1
\subsection{Существующие аналоги}

Среди аналогов разрабатываемой Системы можно отметить «LiveJournal», «Лайфхакер» и «Blogger».
К недостаткам можно отнести то, что данные порталы являются зарубежными, что при текущих обстоятельствах ставит под вопрос стабильность работы на территории РФ.

Отличием данной разработки является сфокусированность на рускоязычную аудиторию.
К преимуществам можно отнести возможность предоставления доступа к истории прочитанных публикаций, а также стабильность работы на территории России.

%% Т. И. Вишневская, Т. Н. Романова
%% Практикум по разработке распределённых систем обработки информации
%% Данный раздел в перечислен в 3.2.1
\subsection*{Описание системы}
%%  — основное назначение системы;
%%  — область её использования.

Система должна представлять собой публичную площадку в сети Интернет для обмена информацией посредством ведения онлайн-дневников и обсуждения прочитанного в комментариях к публикациям.
Пользователь волен публиковать собственные записи как автор или подписываться на других пользователей, а также на категории (теги) для формирования своей ленты публикаций на прочтение, оценивать публикации и комментарии, просматривать самые популярные публикации за всё время.

%Система должна предоставлять пользователю возможность вести собственный онлайн-дневник, читать дневники других пользователей, подписываться на интересующих авторов, а также обмениваться мнениями в комментариях к статьям.
%У пользователей должна быть возможность оценивать статьи и комментарии, сортировать их по дате публикации и по рейтингу.
%Система должна позволять фильтровать статьи по определённым тегам, а также формировать топ-лист самых читаемых статей за всё время.

\section{Основания для разработки}

%% ГОСТ 19.201-78
%% 2.2. В разделе "Основание для разработки" должны быть указаны:
%%    — документ (документы), на основании которых ведётся разработка;
%%    — организация, утвердившая этот документ, и дата его утверждения;
%%    — наименование и (или) условное обозначение темы разработки.

%% Т. И. Вишневская, Т. Н. Романова
%% Лекции по технологиям программирования 2011
%% 2. Основания для разработки.
%% «На основании приказа такого-то по корпорации», или «на основании учебного плана кафедры ИУ7, разработать техническое задание и программное обеспечение в рамках учебного курса распределённые системы».

Разработка ведётся в рамках выполнения лабораторных работ по курсу «Методология программной инженерии» на основании задания на курсовой проект по дисциплине «Распределённые системы обработки информации» на кафедре «Программное обеспечение ЭВМ и информационные технологии» факультета «Информатика, искусственный интеллект и системы управления» федерального государственного бюджетного образовательного учреждения высшего образования «Московский государственный технический университет имени Н. Э. Баумана (национальный исследовательский университет)».

\section{Назначение разработки}

%% ГОСТ 19.201-78
%% 2.3. В разделе "Назначение разработки" должно быть указано функциональное и эксплуатационное назначение программы или программного изделия.

%% Т. И. Вишневская, Т. Н. Романова
%% Лекции по технологиям программирования 2011
%% 3. Назначение разработки
%% Необходимо веско аргументировать назначение программы.
%% Всё это обосновать детально.
%% «Аналогичные разработки существуют, но они обладают рядом
%% недостатков», «проанализировав предметную область, я выявил слабые места:
%%   1. Ручной труд,
%%   2. Неэффективно работает существующая система,
%%   3. Избыточная функциональность существующей системы,
%%   4. 5. 6. ... другие недостатки,
%%   7. Необходимо добавить конкретную функциональность в рамках сущ. системы»

Главное назначение разрабатываемой Системы — предоставление пользователю возможности ведения собственного онлайн-дневника; просмотра публикаций других пользователей; формирования личного круга интересов посредством подписок на определённых авторов и теги; обсуждения публикаций в комментариях к ним; просмотра истории прочитанных публикаций.
У пользователей должна быть возможность оценивать публикации и комментарии, сортировать их по дате и рейтингу.
Система должна позволять фильтровать публикации по определённым тегам, а также формировать топ-лист самых популярных публикаций за всё время.

\section{Требования к программе} % или программному изделию

%% Т. И. Вишневская, Т. Н. Романова
%% Практикум по разработке распределённых систем обработки информации
%% Данный раздел в перечислен в 3.2.1
\subsection{Общие требования к системе}
%%   — по модернизации и восстановлению системы;
%%   — по безопасности системы.

%% Т. И. Вишневская, Т. Н. Романова
%% Лекции по технологиям программирования 2011
%% 4.1. Сначала высокоуровневые требования — это требования, которые касаются абсолютно всех подсистем, которые я интегрирую.
%% Здесь можно сослаться на политические, юридические или финансовые документы, на основе которых ПО будет функционировать.
%% Режим:. Напр., хочу разработать систему, которая должна будет работать круглый год 24 часа в сутки. Пишут: «режим функционирования системы 24/7/365».
%% Или «ежедневно», «ежечастно», «только по ночам» и т. д.

\begin{itemize}
	\item Разрабатываемое ПО должно обеспечивать функционирование Системы в режиме 24/7/365 со среднеговодым временем доступности не менее 99,9 \%.
	Допустимое время, в течение которого Система недоступна, за год должно составлять менее $24 \cdot 365 \cdot 0,001 = 8,76$ часа.
	\item Время восстановления Системы после сбоя не должно превышать 15 минут.
	\item Каждый узел должен автоматически восстанавливаться после сбоя.
	\item Обеспечить безопасность работоспособности Системы за счёт отказоустойчивости узлов.
\end{itemize}

\subsection{Требования к функциональным характеристикам}

%% ГОСТ 19.201-78
%% 2.4.1. В подразделе "Требования к функциональным характеристикам" должны быть указаны требования к составу выполняемых функций, организации входных и выходных данных, временным характеристикам и т. п.

%% Т. И. Вишневская, Т. Н. Романова
%% Практикум по разработке распределённых систем обработки информации
%%   — числовые значения функциональных характеристик (времени отклика);
%%   — учёт латентности.

%% Т. И. Вишневская, Т. Н. Романова
%% Лекции по технологиям программирования 2011
%% 4.3. Требования к функциональным характеристикам.
%% Это «какие показатели» она будет выдавать с точки зрения реактивности системы — время реакции на запрос пользователя.
%% Напр., «одновременно может работать NN клиентов.
%% Если количество клиентов больше, то время отклика увеличивается».

\begin{itemize}
	\item Медиана времени отклика Системы на запросы пользователя на получение информации не должна превышать 3 секунд без учёта латентности географического расположения узла.
	\item Медиана времени отклика Системы на запросы, добавляющие или изменяющие информацию в Системе, не должна превышать 5 секунд без учёта латентности географического расположения узла.
	\item Система должна обеспечивать возможность запуска в современных браузерах: не менее 94,01 \% пользователей Интернета должны иметь возможность пользоваться порталом без какой-либо деградации функционала.
\end{itemize}

%% Т. И. Вишневская, Т. Н. Романова
%% Практикум по разработке распределённых систем обработки информации
%% Данный раздел в перечислен в 3.2.1
\subsection{Функциональные требования} % к порталу с точки зрения пользователя}
%%   — возможные роли пользователей;
%%   — функции пользователей с учётом их роли;
%%   — входные и выходные данные системы.

%% Т. И. Вишневская, Т. Н. Романова
%% Лекции по технологиям программирования 2011
%% 4.2. Функциональные требования.
%% 4.2.1. Ф. Т. с точки зрения пользователя.
%% Требования будущих функций системы с точки зрения пользователя.
%% Напр., «система должна предоставить возможность пользователю просмотреть список заказов», и т. п.
%% Короче, всё, что может сделать система для пользователя.
%%
%% 4.2.2. Ф. Т. с точки зрения администратора.
%% У него полномочий больше.
%% Администратор может не только читать, но и, в отличае от пользователя, изменять и т. п.
%% Перечисляем всё, что может сделать администратор в рамках системы.
%% Админ и юзер – разные роли.
%% Если есть другие роли (напр., менеджер) — то и его описываю.

Система должна:
\begin{enumerate}
	\item обеспечивать
	\begin{enumerate}
		\item регистрацию и авторизацию пользователей с валидацией вводимых данных как через интерфейс приложения;
		\item аутентификацию пользователей;
		\item разделение пользователей на три роли:
		\begin{itemize}
			\item Гость (неавторизованный Пользователь),
			\item Пользователь,
			\item Администратор;
		\end{itemize}
		причём Пользователю доступны все функции Гостя, Администратору — все функции Пользователя;
	\end{enumerate}
	\item предоставлять Гостю следующие функции:
	\begin{enumerate}
		\item получение списка публикаций
		\begin{itemize}
			\item по автору,
			\item по тегу,
			\item всех авторов;
		\end{itemize}
		\item просмотр полного текста публикаций и комментариев к ней;
		\item сортировка публикаций/комментариев по дате/рейтингу;
	\end{enumerate}
	\item предоставлять Пользователю следующие функции:
	\begin{enumerate}
		\item добавление публикаций в свой онлайн-дневник;
		\item редактирование публикаций из своего онлайн-дневника;
		\item удаление публикаций из своего онлайн-дневника;
		\item управление подписками на других авторов;
		\item управление подписками на теги;
		\item получение списка публикаций
		\begin{itemize}
			\item по авторам и тегам, на которые подписан Пользователь;
			\item просмотренных;
		\end{itemize}
		\item добавление комментария к публикации;
		\item редактирование своего комментария;
		\item удаление своего комментария;
		\item оценивать «плюсом» или «минусом» публикации/комментарии;
	\end{enumerate}
	\item предоставлять Администратору следующие функции:
	\begin{itemize}
		\item блокирование части публикации, комментария или пользователя;
		\item отображение статистики просмотров публикаций (см.~4.3.2~пункт~г).
	\end{itemize}
\end{enumerate}

\subsubsection{Требования к организации входных данных}

Входные данные Системы:
\begin{enumerate}
	\item пользователь:
	\begin{itemize}
		\item ФИО — строка, не более 256 символов;
		\item адрес электронный почты — строка, не более 256 символов;
		\item пароль — строка, не менее 8 и не более 256 символов.
	\end{itemize}
	\item Публикация:
	\begin{itemize}
		\item ФИО автора — строка, не более 256 символов;
		\item заголовок — строка, не более 256 символов;
		\item тело — строка, не более 65536 символов;
		\item теги — список из не более 32 строк, каждая из которых состоит из не более 32 символов;
		\item рейтинг — целое число.
	\end{itemize}
	\item Комментарий:
	\begin{itemize}
		\item ФИО автора — строка, не более 256 символов;
		\item текст — строка, не более 4096 символов.
	\end{itemize}
	\item Тег:
	\begin{itemize}
		\item название категории — строка, не более 32 символов.
	\end{itemize}
	%\item Запись статистики:
	%\begin{itemize}
	%	\item объект просмотра:
	%	\begin{itemize}
	%		\item тип объекта (Публикация/Комментарий) — строка, не более 16 символов;
	%		\item идентификатор объекта — UUID4;
	%	\end{itemize}
	%	\item субъект просмотра:
	%	\begin{itemize}
	%		\item ФИО Пользователя — строка, не более 256 символов;
	%		\item адрес электронной почты Пользователя — строка, не более 256 символов.
	%	\end{itemize}
	%\end{itemize}
\end{enumerate}

\subsubsection{Требования к организации выходных данных}

Выходными данными Системы являются веб-страницы.
В зависимости от запроса Пользователя они могут содержать следующие сведения:

\begin{enumerate}
	%\item Пользователь:
	%\begin{itemize}
	%	\item ФИО — строка, не более 256 символов;
	%	\item адрес электронный почты — строка, не более 256 символов;
	%\end{itemize}
	\item Публикация:
	\begin{itemize}
		\item ФИО автора — строка, не более 256 символов;
		\item заголовок — строка, не более 256 символов;
		\item тело — строка, не более 65536 символов;
		\item теги — список из не более 32 строк, каждая из которых состоит из не более 32 символов;
		\item рейтинг — целое число;
		\item дата публикации (генерируется автоматически) — строка, соответствующая формату стандартного вывода POSIX-утилиты date;
		\item идентификатор (генерируется автоматически) — UUID4.
	\end{itemize}
	\item Комментарий:
	\begin{itemize}
		\item ФИО автора — строка, не более 256 символов;
		\item текст — строка, не более 4096 символов.
		\item дата (генерируется автоматически) — строка, соответствующая формату стандартного вывода POSIX-утилиты date;
		\item идентификатор (генерируется автоматически) — UUID4.
	\end{itemize}
	\item Тег:
	\begin{itemize}
		\item название категории — строка, не более 32 символов.
	\end{itemize}
	\item Статистика просмотров публикации:
	\begin{itemize}
		\item всё из пункта а)
		\item таблица с атрибутами:
		\begin{itemize}
			\item ФИО Пользователя — строка, не более 256 символов;
			\item адрес электронной почты Пользователя — строка, не более 256 символов.
			\item дата просмотра (генерируется автоматически) — строка, соответствующая формату стандартного вывода POSIX-утилиты date.
		\end{itemize}
	\end{itemize}
\end{enumerate}

%% Т. И. Вишневская, Т. Н. Романова
%% Практикум по разработке распределённых систем обработки информации
%% Данный раздел в перечислен в 3.2.1
\subsection{Требования к программной реализации}
%% В этом разделе необходимо указать требования согласно п. 1.3 задания по практике.

\begin{enumerate}
	\item Требуется использовать СОА для реализации Системы.
	\item Каждый сервис реализует свою функциональность и взаимодействует с другими сервисами по протоколу HTTP (придерживаться нотации RESTful), либо через очередь.
	\item Каждый сервис при необходимости имеет доступ к связанной с ним базе данных, но не должен иметь доступ к базам данных других сервисов.
	\item Предусмотреть ситуацию недоступности систем, обработку таймаутов и ошибок сервисов.
	В случае ошибки/недоступности некритичного функционала выполнять деградацию функциональности.
	\item Необходимо предусмотреть авторизацию пользователей через
	интерфейс приложения.
	\item  Для запросов, выполняющих обновление данных на нескольких узлах распределенной системы, в случае недоступности одной из систем, необходимо выполнять полный откат транзакции.
	\item Приложение должно поддерживать возможность горизонтального и вертикального масштабирования за счёт увеличения количества функционирующих узлов и совершенствования технологий реализации компонентов и всей архитектуры системы.
\end{enumerate}

\subsection{Топология системы}

Топология разрабатываемой системы изображена на рисунке \ref{img:topology}.

\imght{width=\linewidth}{topology}{Топология системы}

Разрабатываемая система состоит из фронтенда и 5 подсистем:
\begin{itemize}
	\item Сервис публикаций;
	\item Сервис подписок;
	\item Сервис статистики;
	\item Сервис регистрации и авторизации;
	\item Сервис-координатор.
\end{itemize}

\subsubsection{Общие требования к подсистемам}

\begin{enumerate}
	\item Фронтенд – серверное приложение, при разработке которого
	следует учесть следующие нюансы:
	\begin{enumerate}
		\item Фронтенд должен принимать запросы по протоколу HTTP и формировать ответ пользователю в формате HTML-страниц (использование CSS обязательно);
		\item Фронтенд является посредником между пользователями, передавая их запросы последовательно на сервис агрегации запросов;
		\item Запросы от фронтенда могут быть только к сервису-координатору либо сервису регистрации и авторизации для получения токена.
	\end{enumerate}
	\item К реализации бэкендов должны быть предъявлены следующие
	требования:
	\begin{enumerate}
		\item Приём и возврат данных должен происходить в формате JSON по протоколу HTTP;
		\item Все запросы, кроме авторизации пользователя, проходят через сервис-координатор.
		\item Если результаты работы сервиса необходимо сохранять в базе данных, то доступ к ней должен осуществляться по протоколу HTTP.
		Доступ к базе данных может осуществляться только из
		подсистем, работающих напрямую с данными её таблиц.
	\end{enumerate}
\end{enumerate}

\subsubsection{Функциональные требования к сервисам}

\begin{enumerate}
	\item Сервис публикаций — отвечает за добавление публикаций и хранение информации о них.
	В базе данных, ассоциированной с сервисом, должны храниться сущности:
	\begin{itemize}
		\item Публикация, с обязательными полями
		\begin{itemize}
			\item идентификатор;
			\item автор;
			\item заголовок;
			\item тело публикации;
			\item дата добавления;
			\item рейтинг;
		\end{itemize}
		\item Комментарий, с обязательными полями:
		\begin{itemize}
			\item идентификатор;
			\item автор;
			\item текст комментария;
			\item дата добавления;
			\item рейтинг;
		\end{itemize}
		\item Тег, с обязательным полем
		\begin{itemize}
			\item название категории.
		\end{itemize}
	\end{itemize}
	Сервис должен реализовывать следующий функционал:
	\begin{enumerate}
		\item добавление/изменение/удаление публикации;
		\item добавление/изменение/удаление комментария;
		\item добавление/изменение/удаление тега (для Администратора);
		\item получение публикаций по автору/тегу (или всех), с сортировкой по дате или рейтингу;
		\item получение отдельной публикации и комментариев к ней.
	\end{enumerate}
	\item Сервис подписок — осуществляет формирование персонального информационного окружения пользователя.
	Хранимая в базе данных сущность, ассоциированная с сервисом, имеет следующие обязательные поля:
	\begin{itemize}
		\item идентификатор пользователя;
		\item идентификатор автора или тега, на который подписывается читатель.
	\end{itemize}
	Сервис должен реализовывать следующий функционал:
	\begin{enumerate}
		\item добавление/удаление подписки;
		\item получение идентификаторов авторов или тегов, на которые подписан пользователь;
		\item получение всех подписок пользователя.
	\end{enumerate}
	\item Сервис статистики должен реализовывать следующий функционал:
	\begin{itemize}
		\item получение статистики просмотров публикации;
		%\item получение статистики изменения рейтинга публикации или комментария;
	\end{itemize}
	\item Сервис регистрации и авторизации.
	Хранимая в базе данных сущность, ассоциированная с сервисом, имеет следующие обязательные поля:
	\begin{itemize}
		\item ФИО;
		\item адрес электронной почты;
		\item пароль;
	\end{itemize}
	Сервис должен реализовывать следующий функционал:
	\begin{enumerate}
		\item проверка существования пользователя;
		\item регистрация пользователя;
		\item аутентификация пользователя;
		\item удаление пользователя.
	\end{enumerate}
	\item Сервис-координатор должен реализовать диспетчеризацию запросов.
\end{enumerate}

\subsection{Требования к надёжности}

%% ГОСТ 19.201-78
%% 2.4.2. В подразделе "Требования к надёжности" должны быть указаны требования к обеспечению надёжного функционирования (обеспечение устойчивого функционирования, контроль входной и выходной информации, время восстановления после отказа и т. п.).

%% Т. И. Вишневская, Т. Н. Романова
%% Лекции по технологиям программирования 2011
%% 4.4. Надёжность.
%% Следует указать уровень надёжности, который обязан быть у системы, и время восстановления системы после сбоя.
%% Здесь следует описать, как производится контроль входной и выходной информации.
%% Иногда этим занимается специальный блок контроля.
%% Создание резервных копий.

Надёжное (устойчивое) функционирование программы должно быть
обеспечено выполнением совокупности организационно-технических
мероприятий, перечень которых приведён ниже:
\begin{itemize}
	\item организация бесперебойного питания технических средств;
	\item использование лицензионного программного обеспечения;
	\item регулярным выполнением ГОСТ 51188-98. Защита информации.
	Испытания программных средств на наличие компьютерных вирусов.
\end{itemize}

В ситуации недоступности систем, выводится соответствующее сообщение об ошибке.
В случае ошибки/недоступности некритичного функционала, выполняется деградация функциональности.

% \subsection{Условия эксплуатации}

%% ГОСТ 19.201-78
%% 2.4.3. В подразделе "Условия эксплуатации" должны быть указаны условия эксплуатации (температура окружающего воздуха, относительная влажность и т. п. для выбранных типов носителей данных), при которых должны обеспечиваться заданные характеристики, а также вид обслуживания, необходимое количество и квалификация персонала.

%% Т. И. Вишневская, Т. Н. Романова
%% Лекции по технологиям программирования 2011
%% 4.5. Условия эксплуатации.
%% «Температура, влажность» и т. п.
%% Напр., «система должна эксплуатироваться в нестандартных условиях (особые условия, экстремальные):».
%% Т. е. это те технические характеристики, которые должна учитывать программа в своей работе.
%% Этот пункт часто не пишут.
%% Пишут его только в программно-техническом комплексе.

% \subsection{Требования к составу и параметрам технических средств}

%% ГОСТ 19.201-78
%% 2.4.4. В подразделе "Требования к составу и параметрам технических средств" указывают необходимый состав технических средств с указанием их основных технических характеристик.

%% Т. И. Вишневская, Т. Н. Романова
%% Лекции по технологиям программирования 2011
%% 4.6. К составу и параметрам технических средств.
%% Здесь студенты часто делают ошибки.
%% Мы отрабатываем программу на своём ПК (с его конфигурацией).
%% А потом студенты заявляют что-то вроде «прога работает на всём!!», но ведь не факт!
%% Она и медленнее работать будет.
%% Или не совместима с чем-то.
%% Надо протестировать, и только протестировав, мы можем написать «система может работать под.... (и перечисляю всё то, под чем тестировал)».
%% Обычно пишут, указывая минимальные требования и рекомендуемые.
%% «Минимальные требования к технической платформе и операционному окружению».
%% «Техническая платформа: (прописываю всё детально)».
%% «Операционное окружение: (какое ПО, какие платформы, какие версии и т. п.)».
%% Рекомендуемые – это те требования, на которых будет выполняться время отклика и т. п., т. е. программа будет работать наилучшим образом

% \subsection{Требования к информационной и программной совместимости}

%% ГОСТ 19.201-78
%% 2.4.5 В подразделе "Требования к информационной и программной совместимости" должны быть указаны требования к информационным структурам на входе и выходе и методам решения, исходным кодам, языкам программирования и программным средствам, используемым программой.
%%
%% При необходимости должна обеспечиваться защита информации и программ.

%% Т. И. Вишневская, Т. Н. Романова
%% Лекции по технологиям программирования 2011
%% 4.7. Требования к информационной и программной совместимости.
%% Предположим, что-то надо оптимизировать, и делаем это через Matlab.
%% Тогда надо указать, что мы не сами модуль делали, а использовали «сторонний».
%% Предполагаемые методы решения, язык и среду программирования, а также другие программные средства, которые должны взаимодействовать (и каким образом) с нашими программами.
%% С чем программа не может сосуществовать (конфликты), и т. д.
%%
%% ТЗ – это язык промежуточный между профессиональными программистами и заказчиками.
%% Поэтому не должно быть никаких профессионально-жаргонных слов.
%% «Осуществляем back-up» – не годится.
%% Если их не избежать, то следует создать раздел «Глоссарий», в котором требуется описать слова и их определения.
%% Там следует указывать всю терминологию предметной области.
%%
%% По поводу протоколов.
%% Если их диктует заказчик, то «основания протоколов передачи данных такие-то, по требованию заказчика».
%% Если заказчик выдаёт сценарий работы системы, то пишу: «по требованию заказчика, сценарий работы следующий:».
%% А если я сам что-то определяю, то не пишу, т. к. это не исходные данные, а что-то то, чего разработать следует в ходе выполнения работы.
%%
%% В этом же разделе при необходимости указывают степень защиты.
%% Т. е. нужно ли шифрование, или нет.
%% Каким образом будет осуществляться шифрование.
%% Весь ли трафик шифруется, или часть.
%% В конце привожу список стандартов, на основе которых осуществляется шифрование.
%% А в самом начале писать: «Данное ТЗ разработано на основе ГОСТа (и полное название госта)».

% \subsection{Требования к маркировке и упаковке}

%% ГОСТ 19.201-78
%% 2.4.6. В подразделе "Требования к маркировке и упаковке" в общем случае указывают требования к маркировке программного изделия, варианты и способы упаковки.

% \subsection{Требования к транспортированию и хранению}

%% ГОСТ 19.201-78
%% 2.4.7. В подразделе "Требования к транспортированию и хранению" должны быть указаны для программного изделия условия транспортирования, места хранения, условия хранения, условия складирования, сроки хранения в различных условиях.

% \subsection{Специальные требования}

\section{Требования к программной документации}

%% ГОСТ 19.201-78
%% 2.5а. В разделе "Требования к программной документации" должны быть указаны предварительный состав программной документации и, при необходимости, специальные требования к ней.

%% Т. И. Вишневская, Т. Н. Романова
%% Лекции по технологиям программирования 2011
%% 4.8. Требования к программной документации.
%% Документация бывает технологическая, научная, пользовательская и т. п.
%% Напр., разработали ПО, передаю её, коды и инструкцию.
%% Пользователю передаётся только пользовательская документация, инструкцию по установке, инструкцию по использованию, и описание возможных сбоев с инструкциями по их устранению.
%% Ещё инструкцию для администратора.
%% Все остальные вещи – это моё know-how, передавать это не стоит, за исключением тех случаев, когда они сами хотят дорабатывать его и т. д.

Исполнитель должен подготовить и передать Заказчику следующие документы:
\begin{itemize}
	\item руководство администратора Системы;
	\item руководство пользователя Системы.
\end{itemize}

% \section{Технико-экономические показатели}

%% ГОСТ 19.201-78
%% 2.5. В разделе "Технико-экономические показатели" должны быть указаны: ориентировочная экономическая эффективность, предполагаемая годовая потребность, экономические преимущества разработки по сравнению с лучшими отечественными и зарубежными образцами или аналогами.

% \section{Стадии и этапы разработки}

%% ГОСТ 19.201-78
%% 2.6. В разделе "Стадии и этапы разработки" устанавливают необходимые стадии разработки, этапы и содержание работ (перечень программных документов, которые должны быть разработаны, согласованы и утверждены), а также, как правило, сроки разработки и определяют исполнителей.

% \section{Порядок контроля и приёмки}

%% ГОСТ 19.201-78
%% 2.7. В разделе "Порядок контроля и приемки" должны быть указаны виды испытаний и общие требования к приемке работы.

% \section{Приложение А}

%% ГОСТ 19.201-78
%% 2.8. В приложениях к техническому заданию, при необходимости, приводят:
%%    — перечень научно-исследовательских и других работ, обосновывающих разработку;
%%    — схемы алгоритмов, таблицы, описания, обоснования, расчёты и другие документы, которые могут быть использованы при разработке;
%%    — другие источники разработки.

\chapter*{Конструкторский раздел}

\section*{Концептуальный дизайн}

Концептуальный дизайн системы содержит наиболее общие схемы описания функционала приложения с точки зрения пользователей.
Одной из таких схем является IDEF0-модель и графические модели, входящие в неё.
На рисунке \ref{img:idef0-A0}  отображена контекстная диаграмма верхнего уровня, которая обеспечивает наиболее общее или абстрактное описание работы системы.
Данный вид диаграммы позволяет формализовать описание запросов пользователя и ответов системы на данные запросы, отобразив систему в виде «чёрного» ящика.

\imgH{width=\linewidth}{idef0-A0}{Концептуальная модель системы в нотации IDEF0}

Для уточнения деталей работы системы применяется декомпозиция функций, отображённых на диаграмме верхнего уровня, при помощи создания дочерних диаграмм.
В качестве примера на рисунке \ref{img:idef0-A1-A5} изображена дочерняя диаграмма, которая определяет последовательность выполнения операций в системе при обработке запроса пользователя на получение информационного контента.

\imgH{width=\linewidth}{idef0-A1-A5}{Детализированная концептуальная модель системы в нотации IDEF0}

Детализированная диаграмма, создаваемая при декомпозиции, охватывает ту же область, что и родительский блок, но описывает ее более подробно.
Поэтому такая диаграмма может быть создана для любого из запросов, отображённых на диаграмме верхнего уровня.
Каждый из блоков детализированной диаграммы может быть в свою очередь также описан при помощи дочерней диаграммы.

\section*{Сценарии функционирования системы}

Сценарии функционирования или использования системы описывают конкретную последовательность действий, иллюстрирующую поведение пользователя при работе с приложением.
Далее приведены подробные сценарии основных возможных действий пользователя.

\subsection*{Регистрация пользователя}

\begin{enumerate}
	\item Пользователь нажимает на кнопку «Войти» в интерфейсе приложения.
	\item Система перенаправляет пользователя на страницу авторизации, которая содержит поля для заполнения его данных.
	\item Пользователь вводит данные в форму и для завершения регистрации нажимает на кнопку «Регистрация», тем самым подтверждая верность своих данных, а также согласие на их обработку и хранение.
	\item Если пользователь с введённым для регистрации именем уже существует, то
	пользователь перенаправляется на страницу ошибки.
	При успешной регистрации система перенаправляет пользователя на страницу своего профиля.
\end{enumerate}

\subsection*{Авторизация пользователя}

\begin{enumerate}
	\item Пользователь нажимает на кнопку «Войти» в интерфейсе приложения.
	\item Система перенаправляет пользователя на страницу авторизации, которая содержит поля для заполнения его логина и пароля.
	\item Пользователь завершает работу с формой авторизации нажатием кнопки «Войти».
	\item При обнаружении ошибки в данных, система перенаправляет пользователя на страницу ошибки; при совпадении данных с записью в базе данных аккаунтов пользователь получает доступ к системе.
\end{enumerate}

\subsection*{Получение списка подписок}

\begin{enumerate}
	\item Авторизованный пользователь нажимает на кнопку «Подписки».
	\item Система перенаправляет пользователя на страницу, содержащую список пользователей и категорий, на которые он подписан.
\end{enumerate}

\subsection*{Просмотр ранжированного списка публикаций на основе подписок пользователя}

\begin{enumerate}
	\item Авторизованный пользователь нажимает на кнопку «Лента».
	\item Система перенаправляет пользователя на страницу, которая содержит список публикаций, ранжированных по дате создания.
\end{enumerate}

\subsection*{Получение списка просмотренных публикаций}

\begin{enumerate}
	\item Авторизованный пользователь нажимает на кнопку «История просмотров».
	\item Система перенаправляет пользователя на страницу, содержащую список просмотренных пользователем статей, ранжированных по дате просмотра.
\end{enumerate}

\subsection*{Получение статистики}

\begin{enumerate}
	\item Администратор нажимает на кнопку «Панель администратора».
	\item Система перенаправляет администратора на страницу просмотра статистики.
\end{enumerate}

\section*{Диаграммы прецедентов}

Графически сценарии функционирования системы можно представить при помощи диаграмм прецедентов.
Они позволяют схематично отобразить типичные сценарии взаимодействия между клиентами и приложением.
В системе выделены 3 основных роли: Гость, Пользователь и Администратор, диаграммы прецедентов для этих ролей изображены на рисунках \ref{img:use-case-diagram-guest}, \ref{img:use-case-diagram-user} и \ref{img:use-case-diagram-admin}.

\imgH{scale=1.062}{use-case-diagram-guest}{Диаграмма прецедентов с точки зрения Гостя}

\imgH{scale=1.062}{use-case-diagram-user}{Диаграмма прецедентов с точки зрения Пользователя}

\imgH{scale=1.062}{use-case-diagram-admin}{Диаграмма прецедентов с точки зрения Администратора}

\section*{Спецификации сценариев}

Приведённые сценарии могут иметь как основной поток выполнения, который выполняется чаще всего, так и альтернативные потоки, описывающие выполнение запроса при отклонении от основного хода сценария.
Все возможные ходы выполнения сценария описываются при помощи спецификаций.
Примеры спецификаций для описанных выше сценариев приведены в данном разделе.

\def\wA{0.275\textwidth}
\def\wB{0.29\textwidth}
\def\wC{0.365\textwidth}
\begin{table}[H]
	\caption{Спецификация сценария «Регистрация»}
	\small
	\begin{tabular}{|p{\wA}|p{\wB}p{\wC}|}
		\hline
		\multicolumn{1}{|c|}{\multirow{2}{*}{\TableHeader{\wA}{Действие пользователя}}} &
		\multicolumn{2}{c|}{\TableHeader{\wB+\wC}{Отклик системы}}
		\\ \cline{2-3} 
		\multicolumn{1}{|c|}{} &
		\multicolumn{1}{c|}{\TableHeader{\wB}{Нормальный ход сценария}} &
		\multicolumn{1}{c|}{\TableHeader{\wC}{Альтернативный ход сценария}}
		\\ \hline
		\TableData{\wA}{Пользователь нажимает кнопку «Войти»} &
		\multicolumn{1}{l|}{\TableData{\wB}{Открывается страница для ввода данных}} &
		\TableData{\wC}{Открывается страница для ввода данных}
		\\ \hline
		\TableData{\wA}{Пользователь вводит данные в поля и нажимает кнопку «Регистрация»} &
		\multicolumn{1}{l|}{\TableData{\wB}{Открывается страница с профилем созданного пользователя}} &
		\TableData{\wC}{Открывается страница с сообщением об ошибке, что пользователь с такими данными существует}
		\\ \hline
	\end{tabular}
\end{table}

\begin{table}[H]
	\caption{Спецификация сценария «Авторизация»}
	\small
	\begin{tabular}{|p{\wA}|p{\wB}p{\wC}|}
		\hline
		\multicolumn{1}{|c|}{\multirow{2}{*}{\TableHeader{\wA}{Действие пользователя}}} &
		\multicolumn{2}{c|}{\TableHeader{\wB+\wC}{Отклик системы}}
		\\ \cline{2-3} 
		\multicolumn{1}{|c|}{} &
		\multicolumn{1}{c|}{\TableHeader{\wB}{Нормальный ход сценария}} &
		\multicolumn{1}{c|}{\TableHeader{\wC}{Альтернативный ход сценария}}
		\\ \hline
		\TableData{\wA}{Пользователь нажимает кнопку «Войти»} &
		\multicolumn{1}{l|}{\TableData{\wB}{Открывается страница для ввода данных}} &
		\TableData{\wC}{Открывается страница для ввода данных}
		\\ \hline
		\TableData{\wA}{Пользователь вводит данные в поля и нажимает кнопку «Войти»} &
		\multicolumn{1}{l|}{\TableData{\wB}{Открывается страница профиля пользователя}} &
		\TableData{\wC}{Открывается страница с сообщением об ошибке о неверно введённых данных}
		\\ \hline
	\end{tabular}
\end{table}

\def\wA{0.17\textwidth}
\def\wB{0.26\textwidth}
\def\wC{0.50\textwidth}
\begin{table}[H]
	\caption{Спецификация сценария «Просмотр своих подписок»}
	\small
	\begin{tabular}{|p{\wA}|p{\wB}p{\wC}|}
		\hline
		\multicolumn{1}{|c|}{\multirow{2}{*}{\TableHeader{\wA}{Действие пользователя}}} &
		\multicolumn{2}{c|}{\TableHeader{\wB+\wC}{Отклик системы}}
		\\ \cline{2-3} 
		\multicolumn{1}{|c|}{} &
		\multicolumn{1}{c|}{\TableHeader{\wB}{Нормальный ход сценария}} &
		\multicolumn{1}{c|}{\TableHeader{\wC}{Альтернативный ход сценария}}
		\\ \hline
		\TableData{\wA}{Пользователь \\ нажимает \\ кнопку \\ «Подписки»} &
		\multicolumn{1}{l|}{\TableData{\wB}{Открывается страница со списком пользователей и категорий, на которые подписан пользователь}} &
		\TableData{\wC}{Открывается страница с сообщением об отсутствии подписок \\ ИЛИ \\ Открывается страница с сообщением, что сервис подписок недоступен}
		\\ \hline
	\end{tabular}
\end{table}

\begin{table}[H]
	\caption{Спецификация сценария «Просмотр своей ленты информационного контента»}
	\small
	\begin{tabular}{|p{\wA}|p{\wB}p{\wC}|}
		\hline
		\multicolumn{1}{|c|}{\multirow{2}{*}{\TableHeader{\wA}{Действие пользователя}}} &
		\multicolumn{2}{c|}{\TableHeader{\wB+\wC}{Отклик системы}}
		\\ \cline{2-3} 
		\multicolumn{1}{|c|}{} &
		\multicolumn{1}{c|}{\TableHeader{\wB}{Нормальный ход сценария}} &
		\multicolumn{1}{c|}{\TableHeader{\wC}{Альтернативный ход сценария}}
		\\ \hline
		\TableData{\wA}{Пользователь \\ нажимает \\ кнопку \\ «Лента»} &
		\multicolumn{1}{l|}{\TableData{\wB}{Открывается страница со списком публикаций авторов и по категориям, на которые подписан пользователь}} &
		\TableData{\wC}{Открывается страница с сообщением, что у пользователя нет подписок \\ ИЛИ \\ Открывается страница ошибки с сообщением, что сервис публикаций недоступен}
		\\ \hline
	\end{tabular}
\end{table}

\def\wA{0.275\textwidth}
\def\wB{0.29\textwidth}
\def\wC{0.365\textwidth}
\begin{table}[H]
	\caption{Спецификация сценария «История просмотренных публикаций»}
	\small
	\begin{tabular}{|p{\wA}|p{\wB}p{\wC}|}
		\hline
		\multicolumn{1}{|c|}{\multirow{2}{*}{\TableHeader{\wA}{Действие пользователя}}} &
		\multicolumn{2}{c|}{\TableHeader{\wB+\wC}{Отклик системы}}
		\\ \cline{2-3} 
		\multicolumn{1}{|c|}{} &
		\multicolumn{1}{c|}{\TableHeader{\wB}{Нормальный ход сценария}} &
		\multicolumn{1}{c|}{\TableHeader{\wC}{Альтернативный ход сценария}}
		\\ \hline
		\TableData{\wA}{Пользователь нажимает кнопку «История просмотров»} &
		\multicolumn{1}{l|}{\TableData{\wB}{Открывается страница со списком просмотренных публикаций}} &
		\TableData{\wC}{Открывается страница с сообщением, что история пуста}
		\\ \hline
	\end{tabular}
\end{table}

\def\wA{0.30\textwidth}
\def\wB{0.31\textwidth}
\def\wC{0.32\textwidth}
\begin{table}[H]
	\caption{Спецификация сценария «Получение статистики»}
	\small
	\begin{tabular}{|p{\wA}|p{\wB}p{\wC}|}
		\hline
		\multicolumn{1}{|c|}{\multirow{2}{*}{\TableHeader{\wA}{Действие пользователя}}} &
		\multicolumn{2}{c|}{\TableHeader{\wB+\wC}{Отклик системы}}
		\\ \cline{2-3} 
		\multicolumn{1}{|c|}{} &
		\multicolumn{1}{c|}{\TableHeader{\wB}{Нормальный ход сценария}} &
		\multicolumn{1}{c|}{\TableHeader{\wC}{Альтернативный ход сценария}}
		\\ \hline
		\TableData{\wA}{Пользователь авторизуется с правами Администратора} &
		\multicolumn{1}{l|}{\TableData{\wB}{Открывается страница с профилем Администратора}} &
		\TableData{\wC}{Открывается страница с профилем Администратора}
		\\ \hline
		\TableData{\wA}{Пользователь нажимает кнопку «Панель администратора} &
		\multicolumn{1}{l|}{\TableData{\wB}{Открывается страница, содержащая статистику просмотров публикаций}} &
		\TableData{\wC}{Открывается страница ошибки с сообщением, что сервис статистики недоступен}
		\\ \hline
	\end{tabular}
\end{table}
\let\wA\relax
\let\wB\relax
\let\wC\relax

\section*{Логический дизайн}

В процессе создания концептуального дизайна системы были отражены основные сценарии взаимодействия пользователя и системы.
В разделе логического дизайна представлена организация элементов системы и их взаимодействие между собой.
На основе функциональных требований к выделенным подсистемам, а также объектов, о которых необходимо хранить данные в системе, была разработана схема данных приложения.
Результат её проектирования отображён на условной ER-диаграмме, представленной на рисунке \ref{img:erd-chen}.
На данной схеме прямоугольниками обозначены ключевые сущности, а ромбами — связи между ними.
Участие сущности в отношении с другой сущностью отмечается линией, соединяющей их.
Число, располагающееся около линии, означает тип связи между соединёнными сущностями.

\imgH{width=\linewidth}{erd-chen}{ER-диаграмма данных системы}

На следующей стадии проектирования, добавив в схему данных атрибуты сущностей, получаем схему базы данных, которая изображена на рисунке \ref{img:erd-crow's-foot}.

\imgH{width=\linewidth}{erd-crow's-foot}{Схема базы данных системы}

Далее приводятся спецификации таблиц базы данных, приведённых на рисунке \ref{img:erd-crow's-foot}.

\begin{table}[H]
	\caption{Спецификация таблицы User}
	\begin{tabular}{|l|l|l|}
		\hline
		\multicolumn{1}{|c|}{\textbf{Имя атрибута}} & \multicolumn{1}{c|}{\textbf{Тип атрибута}} & \multicolumn{1}{c|}{\textbf{Описание атрибута}} \\ \hline
		id                                          & UUID4                                      & Идентификатор пользователя                      \\ \hline
		email                                       & string                                     & Электронная почта                               \\ \hline
		password\_hash                              & string                                     & Хеш пароля                                      \\ \hline
		full\_name                                  & string                                     & ФИО                                             \\ \hline
		role                                        & string                                     & Роль (Пользователь или Администратор)           \\ \hline
	\end{tabular}
\end{table}

\begin{table}[H]
	\caption{Спецификация таблицы Tag}
	\begin{tabular}{|l|l|l|}
		\hline
		\multicolumn{1}{|c|}{\textbf{Имя атрибута}} & \multicolumn{1}{c|}{\textbf{Тип атрибута}} & \multicolumn{1}{c|}{\textbf{Описание атрибута}} \\ \hline
		id                                          & UUID4                                      & Идентификатор тега                              \\ \hline
		name                                        & string                                     & Категория                                       \\ \hline
	\end{tabular}
\end{table}

\begin{table}[H]
	\caption{Спецификация таблицы Publication}
	\begin{tabular}{|l|l|l|}
		\hline
		\multicolumn{1}{|c|}{\textbf{Имя атрибута}} & \multicolumn{1}{c|}{\textbf{Тип атрибута}} & \multicolumn{1}{c|}{\textbf{Описание атрибута}} \\ \hline
		id                                          & UUID4                                      & Идентификатор публикации                        \\ \hline
		author\_id                                  & UUID4                                      & Идентификатор автора                            \\ \hline
		title                                       & string                                     & Заголовок                                       \\ \hline
		body                                        & string                                     & Тело                                            \\ \hline
		pub\_date                                   & timestamp                                  & Дата публикации                                 \\ \hline
		rating                                      & int                                        & Рейтинг                                         \\ \hline
	\end{tabular}
\end{table}

\begin{table}[H]
	\caption{Спецификация таблицы Comment}
	\begin{tabular}{|l|l|l|}
		\hline
		\multicolumn{1}{|c|}{\textbf{Имя атрибута}} & \multicolumn{1}{c|}{\textbf{Тип атрибута}} & \multicolumn{1}{c|}{\textbf{Описание атрибута}} \\ \hline
		id                                          & UUID4                                      & Идентификатор комментария                       \\ \hline
		author\_id                                  & UUID4                                      & Идентификатор автора                            \\ \hline
		publication\_id                             & UUID4                                      & Идентификатор публикации                        \\ \hline
		body                                        & string                                     & Тело                                            \\ \hline
		pub\_date                                   & timestamp                                  & Дата публикации комментария                     \\ \hline
		rating                                      & int                                        & Рейтинг                                         \\ \hline
	\end{tabular}
\end{table}

\begin{table}[H]
	\caption{Спецификация таблицы Subscription}
	\begin{tabular}{|l|l|l|}
		\hline
		\multicolumn{1}{|c|}{\textbf{Имя атрибута}} & \multicolumn{1}{c|}{\textbf{Тип атрибута}} & \multicolumn{1}{c|}{\textbf{Описание атрибута}} \\ \hline
		id                                          & UUID4                                      & Идентификатор сущности                          \\ \hline
		user\_id                                    & UUID4                                      & Идентификатор пользователя                      \\ \hline
		sub\_id                                     & UUID4                                      & Идентификатор объекта подписки                  \\ \hline
		type                                        & string                                     & Тип подписки (Тег или Автор)                    \\ \hline
	\end{tabular}
\end{table}

\begin{table}[H]
	\caption{Спецификация таблицы History}
	\begin{tabular}{|l|l|l|}
		\hline
		\multicolumn{1}{|c|}{\textbf{Имя атрибута}} & \multicolumn{1}{c|}{\textbf{Тип атрибута}} & \multicolumn{1}{c|}{\textbf{Описание атрибута}} \\ \hline
		id                                          & UUID4                                      & Идентификатор сущности                          \\ \hline
		user\_id                                    & UUID4                                      & Идентификатор пользователя                      \\ \hline
		publication\_id                             & UUID4                                      & Идентификатор публикации                        \\ \hline
		date                                        & string                                     & Дата просмотра                                  \\ \hline
	\end{tabular}
\end{table}

\begin{table}[H]
	\caption{Спецификация таблицы Publication\_tag}
	\begin{tabular}{|l|l|l|}
		\hline
		\multicolumn{1}{|c|}{\textbf{Имя атрибута}} & \multicolumn{1}{c|}{\textbf{Тип атрибута}} & \multicolumn{1}{c|}{\textbf{Описание атрибута}} \\ \hline
		id                                          & UUID4                                      & Идентификатор сущности                          \\ \hline
		publication\_id                             & UUID4                                      & Идентификатор публикации                        \\ \hline
		tag\_id                                     & UUID4                                      & Идентификатор тега                              \\ \hline
	\end{tabular}
\end{table}

\section*{Диаграммы последовательности действий}

Для описания поведения компонентов системы на единой оси времени используются диаграммы последовательности действий, при помощи которых можно описать последовательность действий для каждого прецедента, необходимую для
достижения цели.
Например, на рисунке \ref{img:sequence-diagram} изображён процесс получения списка публикаций пользователем на основе его подписок.

\imgH{width=\linewidth}{sequence-diagram}{Диаграмма последовательности действий при запросе пользователем публикаций на основе его подписок}

Сервис-координатор отправляет запрос сервису подписок на получение списка авторов и тегов, на которые подписан пользователь.
После получения данных сервис-координатор обращается к сервису публикаций, чтобы получить список публикаций по переданным авторам и тегам.
После окончания данной процедуры главный сервис ранжирует список по времени публикации, формирует веб-страницу и возвращает её пользователю.

\section*{Диаграмма потоков данных}

Рассматриваемая система предполагает распределённое хранение данных.
Все данные системы предполагают хранение в единой базе данных, хранилищами данных являются таблицы.
Диаграмма потоков данных, представленная на рисунке \ref{img:dfd}, отображает модель информационной системы с точки зрения хранения, передачи и обработки данных во время обработки запроса пользователя на получение новостей.

\imgH{width=\linewidth}{dfd}{Диаграмма потоков данных при запросе пользователем публикаций на основе его подписок}

\section*{Архитектура системы}

Основополагающей идеей построения программной архитектуры является идея снижения сложности системы путём абстракции и разграничения полномочий.
В данном проекте каждая функциональная область реализована посредством собственного микросервиса.
Этот подход позволяет бороться со сложностью современных систем.
Архитектура системы призвана показать способ развёртывания системы во внешних средах.
На рисунке \ref{img:architecture} представлена архитектура системы, которая показывает размещение элементов системы на физических носителях и способах их взаимодействия, то есть, указаны протоколы, по которым происходит информационный обмен.

\imgH{width=\linewidth}{architecture}{Архитектура разрабатываемой системы}

\chapter*{Технологический раздел}

\section*{Выбор языка и архитектурного стиля взаимодействия компонентов системы}

Для реализации системы использовалась сервис-ориентированная архитектура (СОА).
Архитектурным стилем взаимодействия между сервисами выбран REST (Representational State Transfer — «передача состояния представления»).
В качестве языка реализации серверной части приложения выбран Python, так как он является одним из ведущих языков в разработке информационных систем на сегодняшний день, а также обладает всеми необходимыми особенностями для реализации СОА и REST.

Для системы выбран фреймворк Django.
Он обладает возможностью запуска на разных платформах, использует только необходимые библиотеки и быстро развивается.

Фреймворк использует модель MVC (Model-View-Controller — «Модель-Представление-Контроллер»).
Он содержит три основных компонента:
\begin{itemize}
	\item Модель (model): описывает используемые в приложении данные, а также логику, которая связана непосредственно с данными.
	Как правило, объекты моделей хранятся в базе данных.
	Модель не должна содержать логику взаимодействия с пользователем и не должна определять механизм обработки запроса.
	Кроме того, модель не должна содержать логику отображения данных в представлении.

	\item Представление (view): отвечает за визуальную часть или пользовательский интерфейс, зачастую (как и в этой работе) это html-страница, через которую пользователь взаимодействует с приложением.
	Также представление может содержать логику, связанную с отображением данных.
	Представление не должно содержать логику обработки запроса пользователя или управления данными.

	\item Контроллер (controller): представляет центральный компонент MVC, который обеспечивает связь между пользователем и приложением, представлением и моделью.
	Контроллер содержит логику обработки запроса пользователя.
	Контроллер получает вводимые пользователем данные и обрабатывает их.
	В зависимости от результатов обработки, отправляет пользователю определённый вывод, например, в виде представления, наполненного данными моделей.
\end{itemize}

Данная архитектура позволяет чётко разделять ответственность между компонентами, отделять бизнес-логику от её визуализации.
Например, для добавления поддержки мобильных устройств достаточно добавить только соответствующее представление, не меняя модель и контроллер.

\section*{Выбор СУБД}

Для работы с базой данных Django использует собственный ORM, в котором модель данных описывается классами Python, и по ней генерируется схема базы данных.
Так как, согласно ТЗ, в базе данных не требуется хранить сложные объекты, например, файлы, для проекта подходит реляционная база данных.
Поэтому для хранения данных была выбрана СУБД Postgre SQL.
PostgreSQL — реляционная система управления базами данных.
Она является не-коммерческим ПО с открытым исходным кодом.
Для работы с этой СУБД существуют библиотеки для таких распространённых языков программирования как Python, Ruby, Perl, PHP, C, C++, Java, C\#, Go.
Она работает под управлением многих операционных систем: Linux, MacOS, Windows, FreeBSD, Solaris и многих других.
PostgreSQL поддерживает распределенные транзакции, что позволяет использовать его в проекте для хранения финансовых данных.
По сравнению с MySQL система PostgreSQL лучше работает с репликацией, так как в ней существует WAL журнал (средство восстановления системы в случае сбоя) физической модификации страниц.
PostgreSQL осуществляет асинхронную репликацию типа «ведущий – ведомый», также возможна репликация формата «мастер-мастер».

\section{Обеспечение масштабируемости}

Как было отмечено выше, СОА позволяет масштабировать систему горизонтально с использованием сервисов-балансировщиков.
Совместное использование СОА и REST подходов, предоставляет возможность с лёгкостью добавлять и удалять сервисы, динамически распределяя нагрузку между существующими.
Данную функциональность поддерживает, например, веб-сервер nginx.


\end{document}
