\documentclass{bmstu-gost-7-32}


\renewcommand*\thesection{\arabic{section}}


\begin{document}

\chapter*{Техническое задание}

%% Т. И. Вишневская, Т. Н. Романова
%% Практикум по разработке распределённых систем обработки информации
%% Данный раздел перечислен в 3.2.1
\section*{Глоссарий}

Тег — идентификатор для категоризации, описания, поиска и задания внутренней структуры.


\section{Введение}

%% ГОСТ 19.201-78
%% 2.1. В разделе "Введение" указывают наименование, краткую характеристику области применения программы или программного изделия и объекта, в котором используют программу или программное изделие.

\subsection{Наименование программы}

Наименование программы — «Распределённая система ведения онлайн-дневников (блогов)» (далее — Система).

\subsection{Краткое описание предметной области}

% https://moluch.ru/young/archive/15/1158/
В современном мире одну из лидирующих позиций занимает информационное пространство — это публичная площадка в сети Интернет, где человек излагает свои мысли.
Множество людей наблюдает за подобными публичными площадками и людьми, которые их ведут, в различных социальных сетях.

Один из наиболее интересных видов информационного пространства — блог.
Термин «блог» произошёл от английского weblog («logging the web» — записывать события в сеть).
Впервые его использовал американский программист Йори Баргер в 1997 году для обозначения сетевого дневника.

Блог — это личный дневник, который размещается в интернете, имеет яркую индивидуальность, оригинальное содержание и свою аудиторию.
В отличие от реальных дневников, которые читают только сами авторы, записи интернет-дневников принято выкладывать на всеобщее обозрение.

Людей, ведущих блог, называют блогерами.
Блогер — это любой человек, который ведёт собственный электронный дневник и является его администратором.

%% Т. И. Вишневская, Т. Н. Романова
%% Практикум по разработке распределённых систем обработки информации
%% Данный раздел в перечислен в 3.2.1
\subsection{Существующие аналоги}

Среди аналогов разрабатываемой системы можно отметить «LiveJournal», «Лайфхакер» и «Blogger».
Данный проект должен иметь следующие преимущества перед существующими аналогами:
\begin{itemize}
	\item возможность ограничения доступа к содержанию статей для неподписанных на блог читателей;
	\item предоставление доступа к истории прочитанных статей.
\end{itemize}


%% Т. И. Вишневская, Т. Н. Романова
%% Практикум по разработке распределённых систем обработки информации
%% Данный раздел в перечислен в 3.2.1
\subsection{Описание системы}
%%  — основное назначение системы;
%%  — область её использования.

Система должна представлять собой публичную площадку в сети Интернет для обмена информацией посредством ведения онлайн-дневников и обсуждения прочитанного в комментариях к статьям.
Пользователь волен публиковать собственные записи как автор или подписываться на других пользователей, а также на категории (теги) для формирования своей ленты статей на прочтение, оценивать публикации и комментарии, просматривать самые читаемые и популярные статьи за всё время.

%Система должна предоставлять пользователю возможность вести собственный онлайн-дневник, читать дневники других пользователей, подписываться на интересующих авторов, а также обмениваться мнениями в комментариях к статьям.
%У пользователей должна быть возможность оценивать статьи и комментарии, сортировать их по дате публикации и по рейтингу.
%Система должна позволять фильтровать статьи по определённым тегам, а также формировать топ-лист самых читаемых статей за всё время.

\section{Основания для разработки}

%% ГОСТ 19.201-78
%% 2.2. В разделе "Основание для разработки" должны быть указаны:
%%    — документ (документы), на основании которых ведётся разработка;
%%    — организация, утвердившая этот документ, и дата его утверждения;
%%    — наименование и (или) условное обозначение темы разработки.

Разработка ведётся в рамках выполнения курсового проекта по дисциплине «Распределённые системы обработки информации» на кафедре «Программное обеспечение ЭВМ и информационные технологии» факультета «Информатика, искусственный интеллект и системы управления» федерального государственного бюджетного образовательного учреждения высшего образования «Московский государственный технический университет имени Н. Э. Баумана (национальный исследовательский университет)».

\section{Назначение разработки}

%% ГОСТ 19.201-78
%% 2.3. В разделе "Назначение разработки" должно быть указано функциональное и эксплуатационное назначение программы или программного изделия.

Главное назначение разрабатываемой системы — предоставление пользователю возможности ведения собственного онлайн-дневника; просмотра статей других пользователей; формирования личного круга интересов посредством подписок на определённых авторов и теги; обсуждения статей в комментариях к ним; просмотра истории прочитанных статей.
У пользователей должна быть возможность оценивать статьи и комментарии, сортировать их по дате публикации и по рейтингу.
Система должна позволять фильтровать статьи по определённым тегам, а также формировать топ-лист самых читаемых статей за всё время.

\section{Требования к программе} % или программному изделию

%% Т. И. Вишневская, Т. Н. Романова
%% Практикум по разработке распределённых систем обработки информации
%% Данный раздел в перечислен в 3.2.1
\subsection{Общие требования к системе}
%%   — по модернизации и восстановлению системы;
%%   — по безопасности системы.

\subsection{Требования к функциональным характеристикам}

%% ГОСТ 19.201-78
%% 2.4.1. В подразделе "Требования к функциональным характеристикам" должны быть указаны требования к составу выполняемых функций, организации входных и выходных данных, временным характеристикам и т. п.

%% Т. И. Вишневская, Т. Н. Романова
%% Практикум по разработке распределённых систем обработки информации
%%   — числовые значения функциональных характеристик (времени отклика);
%%   — учёт латентности.

%% Т. И. Вишневская, Т. Н. Романова
%% Практикум по разработке распределённых систем обработки информации
%% Данный раздел в перечислен в 3.2.1
\subsection{Функциональные требования к порталу с точки зрения пользователя}
%%   — возможные роли пользователей;
%%   — функции пользователей с учётом их роли;
%%   — входные и выходные данные системы.

%% Т. И. Вишневская, Т. Н. Романова
%% Практикум по разработке распределённых систем обработки информации
%% Данный раздел в перечислен в 3.2.1
\subsection{Требования к программной реализации}
%% В этом разделе необходимо указать требования согласно п. 1.3 задания по практике.

\subsection{Требования к надёжности}

%% ГОСТ 19.201-78
%% 2.4.2. В подразделе "Требования к надёжности" должны быть указаны требования к обеспечению надёжного функционирования (обеспечение устойчивого функционирования, контроль входной и выходной информации, время восстановления после отказа и т. п.).

% \subsection{Условия эксплуатации}

%% ГОСТ 19.201-78
%% 2.4.3. В подразделе "Условия эксплуатации" должны быть указаны условия эксплуатации (температура окружающего воздуха, относительная влажность и т. п. для выбранных типов носителей данных), при которых должны обеспечиваться заданные характеристики, а также вид обслуживания, необходимое количество и квалификация персонала.

% \subsection{Требования к составу и параметрам технических средств}

%% ГОСТ 19.201-78
%% 2.4.4. В подразделе "Требования к составу и параметрам технических средств" указывают необходимый состав технических средств с указанием их основных технических характеристик.

% \subsection{Требования к информационной и программной совместимости}

%% ГОСТ 19.201-78
%% 2.4.5 В подразделе "Требования к информационной и программной совместимости" должны быть указаны требования к информационным структурам на входе и выходе и методам решения, исходным кодам, языкам программирования и программным средствам, используемым программой.
%%
%% При необходимости должна обеспечиваться защита информации и программ.

% \subsection{Требования к маркировке и упаковке}

%% ГОСТ 19.201-78
%% 2.4.6. В подразделе "Требования к маркировке и упаковке" в общем случае указывают требования к маркировке программного изделия, варианты и способы упаковки.

% \subsection{Требования к транспортированию и хранению}

%% ГОСТ 19.201-78
%% 2.4.7. В подразделе "Требования к транспортированию и хранению" должны быть указаны для программного изделия условия транспортирования, места хранения, условия хранения, условия складирования, сроки хранения в различных условиях.

% \subsection{Специальные требования}

\section{Требования к программной документации}

%% ГОСТ 19.201-78
%% 2.5а. В разделе "Требования к программной документации" должны быть указаны предварительный состав программной документации и, при необходимости, специальные требования к ней.

% \section{Технико-экономические показатели}

%% ГОСТ 19.201-78
%% 2.5. В разделе "Технико-экономические показатели" должны быть указаны: ориентировочная экономическая эффективность, предполагаемая годовая потребность, экономические преимущества разработки по сравнению с лучшими отечественными и зарубежными образцами или аналогами.

% \section{Стадии и этапы разработки}

%% ГОСТ 19.201-78
%% 2.6. В разделе "Стадии и этапы разработки" устанавливают необходимые стадии разработки, этапы и содержание работ (перечень программных документов, которые должны быть разработаны, согласованы и утверждены), а также, как правило, сроки разработки и определяют исполнителей.

% \section{Порядок контроля и приёмки}

%% ГОСТ 19.201-78
%% 2.7. В разделе "Порядок контроля и приемки" должны быть указаны виды испытаний и общие требования к приемке работы.

% \section{Приложение А}

%% ГОСТ 19.201-78
%% 2.8. В приложениях к техническому заданию, при необходимости, приводят:
%%    — перечень научно-исследовательских и других работ, обосновывающих разработку;
%%    — схемы алгоритмов, таблицы, описания, обоснования, расчёты и другие документы, которые могут быть использованы при разработке;
%%    — другие источники разработки.

\end{document}
