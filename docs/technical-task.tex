\documentclass{bmstu-gost-7-32}


\renewcommand*\thesection{\arabic{section}}


\begin{document}

\chapter*{Техническое задание}

%% Т. И. Вишневская, Т. Н. Романова
%% Практикум по разработке распределённых систем обработки информации
%% Данный раздел перечислен в 3.2.1
\section*{Глоссарий}

Латентность географического положения — увеличение времени отклика приложения, обуславливаемое географическим положением элементов системы или пользователя.

% https://ctrlok.com/post/web-server-percentiles/
Медиана времени отклика — максимальное время предоставления данных пользователю для каждого из как минимум половины пользователей.

Публикация — документ, доступный для массового использования (ГОСТ~7.0-99).

СОА — Сервис-ориентированная архитектура — модульный подход к разработке ПО, базирующийся на обеспечении удалённого по стандартизированным протоколам использования распределённых, слабо связанных легко заменяемых компонентов (сервисов) со стандартизированными интерфейсами.

Тег — идентификатор для категоризации, описания или поиска.

Узел системы — сервер, соединённый с другими устройствами как часть компьютерной сети.

UUID4 — формат представления уникальных идентификаторов в СОА.


\section{Введение}

%% ГОСТ 19.201-78
%% 2.1. В разделе "Введение" указывают наименование, краткую характеристику области применения программы или программного изделия и объекта, в котором используют программу или программное изделие.

%% Т. И. Вишневская, Т. Н. Романова
%% Лекции по технологиям программирования 2011
%% 1. Введение.
%% Здесь следует писать краткую характеристику предметной области (медицина, торговля, промышленность).
%% Можно даже нарисовать в виде картинки.
%% Есть торговое предприятие.
%% Есть поставщики и заказчики.
%% Тогда на введении, прямо на странице рисуют схематично всё это. Условно обозначаю, кто соучаствует в процессе.
%%
%% И, конечно, какая проблема стоит, какие решения есть (существующие разработки), какие у них недостатки.
%% Напр., устарела и т.п.
%% Т.е. почему возникла необходимость создания нового ПО, в самом общем виде — в том виде, в каком описывает заказчик.
%% Например «большая доля ручного труда, следует автоматизировать, чтобы снизить человеческий фактор ошибок и повысить производительность»

\subsection{Наименование программы}

Наименование программы — «Распределённая система ведения онлайн-дневников (блогов)» (далее — Система).

\subsection{Краткое описание предметной области}

% https://moluch.ru/young/archive/15/1158/
В современном мире одну из лидирующих позиций занимает информационное пространство — это публичная площадка в сети Интернет, где человек излагает свои мысли.
Множество людей наблюдает за подобными публичными площадками и людьми, которые их ведут, в различных социальных сетях.

Один из наиболее интересных видов информационного пространства — блог.
Термин «блог» произошёл от английского weblog («logging the web» — записывать события в сеть).
Впервые его использовал американский программист Йори Баргер в 1997 году для обозначения сетевого дневника.

Блог — это личный дневник, который размещается в интернете, имеет яркую индивидуальность, оригинальное содержание и свою аудиторию.
В отличие от реальных дневников, которые читают только сами авторы, записи интернет-дневников принято выкладывать на всеобщее обозрение.

Людей, ведущих блог, называют блогерами.
Блогер — это любой человек, который ведёт собственный электронный дневник и является его администратором.

%% Т. И. Вишневская, Т. Н. Романова
%% Практикум по разработке распределённых систем обработки информации
%% Данный раздел в перечислен в 3.2.1
\subsection{Существующие аналоги}

Среди аналогов разрабатываемой Системы можно отметить «LiveJournal», «Лайфхакер» и «Blogger».
К недостаткам можно отнести то, что данные порталы являются зарубежными, что при текущих обстоятельствах ставит под вопрос стабильность работы на территории РФ.

Отличием данной разработки является сфокусированность на рускоязычную аудиторию.
К преимуществам можно отнести возможность предоставления доступа к истории прочитанных публикаций, а также стабильность работы на территории России.

%% Т. И. Вишневская, Т. Н. Романова
%% Практикум по разработке распределённых систем обработки информации
%% Данный раздел в перечислен в 3.2.1
\subsection*{Описание системы}
%%  — основное назначение системы;
%%  — область её использования.

Система должна представлять собой публичную площадку в сети Интернет для обмена информацией посредством ведения онлайн-дневников и обсуждения прочитанного в комментариях к публикациям.
Пользователь волен публиковать собственные записи как автор или подписываться на других пользователей, а также на категории (теги) для формирования своей ленты публикаций на прочтение, оценивать публикации и комментарии, просматривать самые популярные публикации за всё время.

%Система должна предоставлять пользователю возможность вести собственный онлайн-дневник, читать дневники других пользователей, подписываться на интересующих авторов, а также обмениваться мнениями в комментариях к статьям.
%У пользователей должна быть возможность оценивать статьи и комментарии, сортировать их по дате публикации и по рейтингу.
%Система должна позволять фильтровать статьи по определённым тегам, а также формировать топ-лист самых читаемых статей за всё время.

\section{Основания для разработки}

%% ГОСТ 19.201-78
%% 2.2. В разделе "Основание для разработки" должны быть указаны:
%%    — документ (документы), на основании которых ведётся разработка;
%%    — организация, утвердившая этот документ, и дата его утверждения;
%%    — наименование и (или) условное обозначение темы разработки.

%% Т. И. Вишневская, Т. Н. Романова
%% Лекции по технологиям программирования 2011
%% 2. Основания для разработки.
%% «На основании приказа такого-то по корпорации», или «на основании учебного плана кафедры ИУ7, разработать техническое задание и программное обеспечение в рамках учебного курса распределённые системы».

Разработка ведётся в рамках выполнения лабораторных работ по курсу «Методология программной инженерии» на основании задания на курсовой проект по дисциплине «Распределённые системы обработки информации» на кафедре «Программное обеспечение ЭВМ и информационные технологии» факультета «Информатика, искусственный интеллект и системы управления» федерального государственного бюджетного образовательного учреждения высшего образования «Московский государственный технический университет имени Н. Э. Баумана (национальный исследовательский университет)».

\section{Назначение разработки}

%% ГОСТ 19.201-78
%% 2.3. В разделе "Назначение разработки" должно быть указано функциональное и эксплуатационное назначение программы или программного изделия.

%% Т. И. Вишневская, Т. Н. Романова
%% Лекции по технологиям программирования 2011
%% 3. Назначение разработки
%% Необходимо веско аргументировать назначение программы.
%% Всё это обосновать детально.
%% «Аналогичные разработки существуют, но они обладают рядом
%% недостатков», «проанализировав предметную область, я выявил слабые места:
%%   1. Ручной труд,
%%   2. Неэффективно работает существующая система,
%%   3. Избыточная функциональность существующей системы,
%%   4. 5. 6. ... другие недостатки,
%%   7. Необходимо добавить конкретную функциональность в рамках сущ. системы»

Главное назначение разрабатываемой Системы — предоставление пользователю возможности ведения собственного онлайн-дневника; просмотра публикаций других пользователей; формирования личного круга интересов посредством подписок на определённых авторов и теги; обсуждения публикаций в комментариях к ним; просмотра истории прочитанных публикаций.
У пользователей должна быть возможность оценивать публикации и комментарии, сортировать их по дате и рейтингу.
Система должна позволять фильтровать публикации по определённым тегам, а также формировать топ-лист самых популярных публикаций за всё время.

\section{Требования к программе} % или программному изделию

%% Т. И. Вишневская, Т. Н. Романова
%% Практикум по разработке распределённых систем обработки информации
%% Данный раздел в перечислен в 3.2.1
\subsection{Общие требования к системе}
%%   — по модернизации и восстановлению системы;
%%   — по безопасности системы.

%% Т. И. Вишневская, Т. Н. Романова
%% Лекции по технологиям программирования 2011
%% 4.1. Сначала высокоуровневые требования — это требования, которые касаются абсолютно всех подсистем, которые я интегрирую.
%% Здесь можно сослаться на политические, юридические или финансовые документы, на основе которых ПО будет функционировать.
%% Режим:. Напр., хочу разработать систему, которая должна будет работать круглый год 24 часа в сутки. Пишут: «режим функционирования системы 24/7/365».
%% Или «ежедневно», «ежечастно», «только по ночам» и т. д.

\begin{itemize}
	\item Разрабатываемое ПО должно обеспечивать функционирование Системы в режиме 24/7/365 со среднеговодым временем доступности не менее 99,9 \%.
	Допустимое время, в течение которого Система недоступна, за год должно составлять менее $24 \cdot 365 \cdot 0,001 = 8,76$ часа.
	\item Время восстановления Системы после сбоя не должно превышать 15 минут.
	\item Каждый узел должен автоматически восстанавливаться после сбоя.
	\item Обеспечить безопасность работоспособности Системы за счёт отказоустойчивости узлов.
\end{itemize}

\subsection{Требования к функциональным характеристикам}

%% ГОСТ 19.201-78
%% 2.4.1. В подразделе "Требования к функциональным характеристикам" должны быть указаны требования к составу выполняемых функций, организации входных и выходных данных, временным характеристикам и т. п.

%% Т. И. Вишневская, Т. Н. Романова
%% Практикум по разработке распределённых систем обработки информации
%%   — числовые значения функциональных характеристик (времени отклика);
%%   — учёт латентности.

%% Т. И. Вишневская, Т. Н. Романова
%% Лекции по технологиям программирования 2011
%% 4.3. Требования к функциональным характеристикам.
%% Это «какие показатели» она будет выдавать с точки зрения реактивности системы — время реакции на запрос пользователя.
%% Напр., «одновременно может работать NN клиентов.
%% Если количество клиентов больше, то время отклика увеличивается».

\begin{itemize}
	\item Медиана времени отклика Системы на запросы пользователя на получение информации не должна превышать 3 секунд без учёта латентности географического расположения узла.
	\item Медиана времени отклика Системы на запросы, добавляющие или изменяющие информацию в Системе, не должна превышать 5 секунд без учёта латентности географического расположения узла.
	\item Система должна обеспечивать возможность запуска в современных браузерах: не менее 94,01 \% пользователей Интернета должны иметь возможность пользоваться порталом без какой-либо деградации функционала.
\end{itemize}

%% Т. И. Вишневская, Т. Н. Романова
%% Практикум по разработке распределённых систем обработки информации
%% Данный раздел в перечислен в 3.2.1
\subsection{Функциональные требования} % к порталу с точки зрения пользователя}
%%   — возможные роли пользователей;
%%   — функции пользователей с учётом их роли;
%%   — входные и выходные данные системы.

%% Т. И. Вишневская, Т. Н. Романова
%% Лекции по технологиям программирования 2011
%% 4.2. Функциональные требования.
%% 4.2.1. Ф. Т. с точки зрения пользователя.
%% Требования будущих функций системы с точки зрения пользователя.
%% Напр., «система должна предоставить возможность пользователю просмотреть список заказов», и т. п.
%% Короче, всё, что может сделать система для пользователя.
%%
%% 4.2.2. Ф. Т. с точки зрения администратора.
%% У него полномочий больше.
%% Администратор может не только читать, но и, в отличае от пользователя, изменять и т. п.
%% Перечисляем всё, что может сделать администратор в рамках системы.
%% Админ и юзер – разные роли.
%% Если есть другие роли (напр., менеджер) — то и его описываю.

Система должна:
\begin{enumerate}
	\item обеспечивать
	\begin{enumerate}
		\item регистрацию и авторизацию пользователей с валидацией вводимых данных как через интерфейс приложения;
		\item аутентификацию пользователей;
		\item разделение пользователей на три роли:
		\begin{itemize}
			\item Гость (неавторизованный Пользователь),
			\item Пользователь,
			\item Администратор;
		\end{itemize}
		причём Пользователю доступны все функции Гостя, Администратору — все функции Пользователя;
	\end{enumerate}
	\item предоставлять Гостю следующие функции:
	\begin{enumerate}
		\item получение списка публикаций
		\begin{itemize}
			\item по автору,
			\item по тегу,
			\item всех авторов;
		\end{itemize}
		\item просмотр полного текста публикаций и комментариев к ней;
		\item сортировка публикаций/комментариев по дате/рейтингу;
	\end{enumerate}
	\item предоставлять Пользователю следующие функции:
	\begin{enumerate}
		\item добавление публикаций в свой онлайн-дневник;
		\item редактирование публикаций из своего онлайн-дневника;
		\item удаление публикаций из своего онлайн-дневника;
		\item управление подписками на других авторов;
		\item управление подписками на теги;
		\item получение списка публикаций
		\begin{itemize}
			\item по авторам и тегам, на которые подписан Пользователь;
			\item просмотренных;
		\end{itemize}
		\item добавление комментария к публикации;
		\item редактирование своего комментария;
		\item удаление своего комментария;
		\item оценивать «плюсом» или «минусом» публикации/комментарии;
	\end{enumerate}
	\item предоставлять Администратору следующие функции:
	\begin{itemize}
		\item отображение статистики просмотров публикаций.
	\end{itemize}
\end{enumerate}

\subsubsection{Требования к организации входных данных}

Входные данные Системы:
\begin{enumerate}
	\item Пользователь:
	\begin{itemize}
		\item ФИО — строка, не более 256 символов;
		\item адрес электронный почты — строка, не более 256 символов;
		\item пароль — строка, не менее 8 и не более 256 символов.
	\end{itemize}
	\item Публикация:
	\begin{itemize}
		\item ФИО автора — строка, не более 256 символов;
		\item заголовок — строка, не более 256 символов;
		\item тело — строка, не более 65536 символов;
		\item теги — список из не более 32 строк, каждая из которых состоит из не более 32 символов;
		\item рейтинг — целое число.
	\end{itemize}
	\item Комментарий:
	\begin{itemize}
		\item ФИО автора — строка, не более 256 символов;
		\item текст — строка, не более 4096 символов.
	\end{itemize}
	\item Тег:
	\begin{itemize}
		\item название категории — строка, не более 32 символов.
	\end{itemize}
	\item Запись статистики:
	\begin{itemize}
		\item объект просмотра:
		\begin{itemize}
			\item тип объекта (Публикация/Комментарий) — строка, не более 16 символов;
			\item идентификатор объекта — UUID4;
		\end{itemize}
		\item субъект просмотра:
		\begin{itemize}
			\item ФИО Пользователя — строка, не более 256 символов;
			\item адрес электронной почты Пользователя — строка, не более 256 символов.
		\end{itemize}
	\end{itemize}
\end{enumerate}

\subsubsection{Требования к организации выходных данных}

Выходными данными Системы являются веб-страницы.
В зависимости от запроса Пользователя они могут содержать следующие сведения:

\begin{enumerate}
	%\item Пользователь:
	%\begin{itemize}
	%	\item ФИО — строка, не более 256 символов;
	%	\item адрес электронный почты — строка, не более 256 символов;
	%\end{itemize}
	\item Публикация:
	\begin{itemize}
		\item ФИО автора — строка, не более 256 символов;
		\item заголовок — строка, не более 256 символов;
		\item тело — строка, не более 65536 символов;
		\item теги — список из не более 32 строк, каждая из которых состоит из не более 32 символов;
		\item рейтинг — целое число;
		\item дата публикации (генерируется автоматически) — строка, соответствующая формату стандартного вывода POSIX-утилиты date;
		\item идентификатор (генерируется автоматически) — UUID4.
	\end{itemize}
	\item Комментарий:
	\begin{itemize}
		\item ФИО автора — строка, не более 256 символов;
		\item текст — строка, не более 4096 символов.
		\item дата (генерируется автоматически) — строка, соответствующая формату стандартного вывода POSIX-утилиты date;
		\item идентификатор (генерируется автоматически) — UUID4.
	\end{itemize}
	\item Тег:
	\begin{itemize}
		\item название категории — строка, не более 32 символов.
	\end{itemize}
	\item Запись статистики:
	\begin{itemize}
		\item объект просмотра:
		\begin{itemize}
			\item тип объекта (Публикация/Комментарий) — строка, не более 16 символов;
			\item идентификатор объекта — UUID4;
		\end{itemize}
		\item субъект просмотра:
		\begin{itemize}
			\item ФИО Пользователя — строка, не более 256 символов;
			\item адрес электронной почты Пользователя — строка, не более 256 символов.
		\end{itemize}
		\item дата просмотра (генерируется автоматически) — строка, соответствующая формату стандартного вывода POSIX-утилиты date.
	\end{itemize}
\end{enumerate}

%% Т. И. Вишневская, Т. Н. Романова
%% Практикум по разработке распределённых систем обработки информации
%% Данный раздел в перечислен в 3.2.1
\subsection{Требования к программной реализации}
%% В этом разделе необходимо указать требования согласно п. 1.3 задания по практике.

\begin{enumerate}
	\item Требуется использовать СОА для реализации Системы.
	\item Каждый сервис реализует свою функциональность и взаимодействует с другими сервисами по протоколу HTTP (придерживаться нотации RESTful), либо через очередь.
	\item Каждый сервис при необходимости имеет доступ к связанной с ним базе данных, но не должен иметь доступ к базам данных других сервисов.
	\item Предусмотреть ситуацию недоступности систем, обработку таймаутов и ошибок сервисов.
	В случае ошибки/недоступности некритичного функционала выполнять деградацию функциональности.
	\item Необходимо предусмотреть авторизацию пользователей через
	интерфейс приложения.
	\item  Для запросов, выполняющих обновление данных на нескольких узлах распределенной системы, в случае недоступности одной из систем, необходимо выполнять полный откат транзакции.
	\item Приложение должно поддерживать возможность горизонтального и вертикального масштабирования за счёт увеличения количества функционирующих узлов и совершенствования технологий реализации компонентов и всей архитектуры системы.
\end{enumerate}

\subsection{Топология системы}

Топология разрабатываемой системы изображена на рисунке \ref{img:topology}.

\imght{width=\linewidth}{topology}{Топология системы}

Разрабатываемая система состоит из фронтенда и 5 подсистем:
\begin{itemize}
	\item Сервис публикаций;
	\item Сервис подписок;
	\item Сервис статистики;
	\item Сервис регистрации и авторизации;
	\item Сервис-координатор.
\end{itemize}

\subsubsection{Общие требования к подсистемам}

\begin{enumerate}
	\item Фронтенд – серверное приложение, при разработке которого
	следует учесть следующие нюансы:
	\begin{enumerate}
		\item Фронтенд должен принимать запросы по протоколу HTTP и формировать ответ пользователю в формате HTML-страниц (использование CSS обязательно);
		\item Фронтенд является посредником между пользователями, передавая их запросы последовательно на сервис агрегации запросов;
		\item Запросы от фронтенда могут быть только к сервису-координатору либо сервису регистрации и авторизации для получения токена.
	\end{enumerate}
	\item К реализации бэкендов должны быть предъявлены следующие
	требования:
	\begin{enumerate}
		\item Приём и возврат данных должен происходить в формате JSON по протоколу HTTP;
		\item Все запросы, кроме авторизации пользователя, проходят через сервис-координатор.
		\item Если результаты работы сервиса необходимо сохранять в базе данных, то доступ к ней должен осуществляться по протоколу HTTP.
		Доступ к базе данных может осуществляться только из
		подсистем, работающих напрямую с данными её таблиц.
	\end{enumerate}
\end{enumerate}

\subsubsection{Функциональные требования к сервисам}

\begin{enumerate}
	\item Сервис публикаций — отвечает за добавление публикаций и хранение информации о них.
	В базе данных, ассоциированной с сервисом, должны храниться сущности:
	\begin{itemize}
		\item Публикация, с обязательными полями
		\begin{itemize}
			\item идентификатор;
			\item автор;
			\item заголовок;
			\item тело публикации;
			\item дата добавления;
			\item рейтинг;
		\end{itemize}
		\item Комментарий, с обязательными полями:
		\begin{itemize}
			\item идентификатор;
			\item автор;
			\item текст комментария;
			\item дата добавления;
			\item рейтинг;
		\end{itemize}
		\item Тег, с обязательным полем
		\begin{itemize}
			\item название категории.
		\end{itemize}
	\end{itemize}
	Сервис должен реализовывать следующий функционал:
	\begin{enumerate}
		\item добавление/изменение/удаление публикации;
		\item добавление/изменение/удаление комментария;
		\item добавление/изменение/удаление тега (для Администратора);
		\item получение публикаций по автору/тегу (или всех), с сортировкой по дате или рейтингу;
		\item получение отдельной публикации и комментариев к ней.
	\end{enumerate}
	\item Сервис подписок — осуществляет формирование персонального информационного окружения пользователя.
	Хранимая в базе данных сущность, ассоциированная с сервисом, имеет следующие обязательные поля:
	\begin{itemize}
		\item идентификатор пользователя;
		\item идентификатор автора или тега, на который подписывается читатель.
	\end{itemize}
	Сервис должен реализовывать следующий функционал:
	\begin{enumerate}
		\item добавление/удаление подписки;
		\item получение идентификаторов авторов или тегов, на которые подписан пользователь;
		\item получение всех подписок пользователя.
	\end{enumerate}
	\item Сервис статистики должен реализовывать следующий функционал:
	\begin{itemize}
		\item получение статистики изменения просмотров публикации;
		%\item получение статистики изменения рейтинга публикации или комментария;
	\end{itemize}
	\item Сервис регистрации и авторизации.
	Хранимая в базе данных сущность, ассоциированная с сервисом, имеет следующие обязательные поля:
	\begin{itemize}
		\item ФИО;
		\item адрес электронной почты;
		\item пароль;
	\end{itemize}
	Сервис должен реализовывать следующий функционал:
	\begin{enumerate}
		\item проверка существования пользователя;
		\item регистрация пользователя;
		\item аутентификация пользователя;
		\item удаление пользователя.
	\end{enumerate}
	\item Сервис-координатор должен реализовать диспетчеризацию запросов.
\end{enumerate}

\subsection{Требования к надёжности}

%% ГОСТ 19.201-78
%% 2.4.2. В подразделе "Требования к надёжности" должны быть указаны требования к обеспечению надёжного функционирования (обеспечение устойчивого функционирования, контроль входной и выходной информации, время восстановления после отказа и т. п.).

%% Т. И. Вишневская, Т. Н. Романова
%% Лекции по технологиям программирования 2011
%% 4.4. Надёжность.
%% Следует указать уровень надёжности, который обязан быть у системы, и время восстановления системы после сбоя.
%% Здесь следует описать, как производится контроль входной и выходной информации.
%% Иногда этим занимается специальный блок контроля.
%% Создание резервных копий.

Надёжное (устойчивое) функционирование программы должно быть
обеспечено выполнением совокупности организационно-технических
мероприятий, перечень которых приведён ниже:
\begin{itemize}
	\item организация бесперебойного питания технических средств;
	\item использование лицензионного программного обеспечения;
	\item регулярным выполнением ГОСТ 51188-98. Защита информации.
	Испытания программных средств на наличие компьютерных вирусов.
\end{itemize}

В ситуации недоступности систем, выводится соответствующее сообщение об ошибке.
В случае ошибки/недоступности некритичного функционала, выполняется деградация функциональности.

% \subsection{Условия эксплуатации}

%% ГОСТ 19.201-78
%% 2.4.3. В подразделе "Условия эксплуатации" должны быть указаны условия эксплуатации (температура окружающего воздуха, относительная влажность и т. п. для выбранных типов носителей данных), при которых должны обеспечиваться заданные характеристики, а также вид обслуживания, необходимое количество и квалификация персонала.

%% Т. И. Вишневская, Т. Н. Романова
%% Лекции по технологиям программирования 2011
%% 4.5. Условия эксплуатации.
%% «Температура, влажность» и т. п.
%% Напр., «система должна эксплуатироваться в нестандартных условиях (особые условия, экстремальные):».
%% Т. е. это те технические характеристики, которые должна учитывать программа в своей работе.
%% Этот пункт часто не пишут.
%% Пишут его только в программно-техническом комплексе.

% \subsection{Требования к составу и параметрам технических средств}

%% ГОСТ 19.201-78
%% 2.4.4. В подразделе "Требования к составу и параметрам технических средств" указывают необходимый состав технических средств с указанием их основных технических характеристик.

%% Т. И. Вишневская, Т. Н. Романова
%% Лекции по технологиям программирования 2011
%% 4.6. К составу и параметрам технических средств.
%% Здесь студенты часто делают ошибки.
%% Мы отрабатываем программу на своём ПК (с его конфигурацией).
%% А потом студенты заявляют что-то вроде «прога работает на всём!!», но ведь не факт!
%% Она и медленнее работать будет.
%% Или не совместима с чем-то.
%% Надо протестировать, и только протестировав, мы можем написать «система может работать под.... (и перечисляю всё то, под чем тестировал)».
%% Обычно пишут, указывая минимальные требования и рекомендуемые.
%% «Минимальные требования к технической платформе и операционному окружению».
%% «Техническая платформа: (прописываю всё детально)».
%% «Операционное окружение: (какое ПО, какие платформы, какие версии и т. п.)».
%% Рекомендуемые – это те требования, на которых будет выполняться время отклика и т. п., т. е. программа будет работать наилучшим образом

% \subsection{Требования к информационной и программной совместимости}

%% ГОСТ 19.201-78
%% 2.4.5 В подразделе "Требования к информационной и программной совместимости" должны быть указаны требования к информационным структурам на входе и выходе и методам решения, исходным кодам, языкам программирования и программным средствам, используемым программой.
%%
%% При необходимости должна обеспечиваться защита информации и программ.

%% Т. И. Вишневская, Т. Н. Романова
%% Лекции по технологиям программирования 2011
%% 4.7. Требования к информационной и программной совместимости.
%% Предположим, что-то надо оптимизировать, и делаем это через Matlab.
%% Тогда надо указать, что мы не сами модуль делали, а использовали «сторонний».
%% Предполагаемые методы решения, язык и среду программирования, а также другие программные средства, которые должны взаимодействовать (и каким образом) с нашими программами.
%% С чем программа не может сосуществовать (конфликты), и т. д.
%%
%% ТЗ – это язык промежуточный между профессиональными программистами и заказчиками.
%% Поэтому не должно быть никаких профессионально-жаргонных слов.
%% «Осуществляем back-up» – не годится.
%% Если их не избежать, то следует создать раздел «Глоссарий», в котором требуется описать слова и их определения.
%% Там следует указывать всю терминологию предметной области.
%%
%% По поводу протоколов.
%% Если их диктует заказчик, то «основания протоколов передачи данных такие-то, по требованию заказчика».
%% Если заказчик выдаёт сценарий работы системы, то пишу: «по требованию заказчика, сценарий работы следующий:».
%% А если я сам что-то определяю, то не пишу, т. к. это не исходные данные, а что-то то, чего разработать следует в ходе выполнения работы.
%%
%% В этом же разделе при необходимости указывают степень защиты.
%% Т. е. нужно ли шифрование, или нет.
%% Каким образом будет осуществляться шифрование.
%% Весь ли трафик шифруется, или часть.
%% В конце привожу список стандартов, на основе которых осуществляется шифрование.
%% А в самом начале писать: «Данное ТЗ разработано на основе ГОСТа (и полное название госта)».

% \subsection{Требования к маркировке и упаковке}

%% ГОСТ 19.201-78
%% 2.4.6. В подразделе "Требования к маркировке и упаковке" в общем случае указывают требования к маркировке программного изделия, варианты и способы упаковки.

% \subsection{Требования к транспортированию и хранению}

%% ГОСТ 19.201-78
%% 2.4.7. В подразделе "Требования к транспортированию и хранению" должны быть указаны для программного изделия условия транспортирования, места хранения, условия хранения, условия складирования, сроки хранения в различных условиях.

% \subsection{Специальные требования}

\section{Требования к программной документации}

%% ГОСТ 19.201-78
%% 2.5а. В разделе "Требования к программной документации" должны быть указаны предварительный состав программной документации и, при необходимости, специальные требования к ней.

%% Т. И. Вишневская, Т. Н. Романова
%% Лекции по технологиям программирования 2011
%% 4.8. Требования к программной документации.
%% Документация бывает технологическая, научная, пользовательская и т. п.
%% Напр., разработали ПО, передаю её, коды и инструкцию.
%% Пользователю передаётся только пользовательская документация, инструкцию по установке, инструкцию по использованию, и описание возможных сбоев с инструкциями по их устранению.
%% Ещё инструкцию для администратора.
%% Все остальные вещи – это моё know-how, передавать это не стоит, за исключением тех случаев, когда они сами хотят дорабатывать его и т. д.

Исполнитель должен подготовить и передать Заказчику следующие документы:
\begin{itemize}
	\item руководство администратора Системы;
	\item руководство пользователя Системы.
\end{itemize}

% \section{Технико-экономические показатели}

%% ГОСТ 19.201-78
%% 2.5. В разделе "Технико-экономические показатели" должны быть указаны: ориентировочная экономическая эффективность, предполагаемая годовая потребность, экономические преимущества разработки по сравнению с лучшими отечественными и зарубежными образцами или аналогами.

% \section{Стадии и этапы разработки}

%% ГОСТ 19.201-78
%% 2.6. В разделе "Стадии и этапы разработки" устанавливают необходимые стадии разработки, этапы и содержание работ (перечень программных документов, которые должны быть разработаны, согласованы и утверждены), а также, как правило, сроки разработки и определяют исполнителей.

% \section{Порядок контроля и приёмки}

%% ГОСТ 19.201-78
%% 2.7. В разделе "Порядок контроля и приемки" должны быть указаны виды испытаний и общие требования к приемке работы.

% \section{Приложение А}

%% ГОСТ 19.201-78
%% 2.8. В приложениях к техническому заданию, при необходимости, приводят:
%%    — перечень научно-исследовательских и других работ, обосновывающих разработку;
%%    — схемы алгоритмов, таблицы, описания, обоснования, расчёты и другие документы, которые могут быть использованы при разработке;
%%    — другие источники разработки.

\end{document}
