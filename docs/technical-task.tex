\documentclass{bmstu-gost-7-32}


\renewcommand*\thesection{\arabic{section}}


\begin{document}

\chapter*{Техническое задание}

\section*{Глоссарий}

%% Т. И. Вишневская, Т. Н. Романова
%% Практикум по разработке распределённых систем обработки информации
%% Данный раздел не перечислен в 3.2.1
\section{Введение}

%% ГОСТ 19.201-78
%% 2.1. В разделе "Введение" указывают наименование, краткую характеристику области применения программы или программного изделия и объекта, в котором используют программу или программное изделие.

\subsection{Наименование программы}

\subsection{Краткое описание предметной области}

%% Т. И. Вишневская, Т. Н. Романова
%% Практикум по разработке распределённых систем обработки информации
%% Данный раздел в перечислен в 3.2.1
\subsection{Существующие аналоги}

%% Т. И. Вишневская, Т. Н. Романова
%% Практикум по разработке распределённых систем обработки информации
%% Данный раздел в перечислен в 3.2.1
\subsection{Описание системы}

\section{Основания для разработки}

%% ГОСТ 19.201-78
%% 2.2. В разделе "Основание для разработки" должны быть указаны:
%%    — документ (документы), на основании которых ведётся разработка;
%%    — организация, утвердившая этот документ, и дата его утверждения;
%%    — наименование и (или) условное обозначение темы разработки.

\subsection*{Основания для проведения разработки}

\subsection*{Наименование и условное обозначение темы разработки}

\section{Назначение разработки}

%% ГОСТ 19.201-78
%% 2.3. В разделе "Назначение разработки" должно быть указано функциональное и эксплуатационное назначение программы или программного изделия.

\section{Требования к программе} % или программному изделию

\subsection{Требования к функциональным характеристикам}

%% ГОСТ 19.201-78
%% 2.4.1. В подразделе "Требования к функциональным характеристикам" должны быть указаны требования к составу выполняемых функций, организации входных и выходных данных, временным характеристикам и т. п.

\subsection{Требования к надёжности}

%% ГОСТ 19.201-78
%% 2.4.2. В подразделе "Требования к надёжности" должны быть указаны требования к обеспечению надёжного функционирования (обеспечение устойчивого функционирования, контроль входной и выходной информации, время восстановления после отказа и т. п.).

\subsection{Условия эксплуатации}

%% ГОСТ 19.201-78
%% 2.4.3. В подразделе "Условия эксплуатации" должны быть указаны условия эксплуатации (температура окружающего воздуха, относительная влажность и т. п. для выбранных типов носителей данных), при которых должны обеспечиваться заданные характеристики, а также вид обслуживания, необходимое количество и квалификация персонала.

\subsection{Требования к составу и параметрам технических средств}

%% ГОСТ 19.201-78
%% 2.4.4. В подразделе "Требования к составу и параметрам технических средств" указывают необходимый состав технических средств с указанием их основных технических характеристик.

\subsection{Требования к информационной и программной совместимости}

%% ГОСТ 19.201-78
%% 2.4.5 В подразделе "Требования к информационной и программной совместимости" должны быть указаны требования к информационным структурам на входе и выходе и методам решения, исходным кодам, языкам программирования и программным средствам, используемым программой.
%%
%% При необходимости должна обеспечиваться защита информации и программ.

\subsection{Требования к маркировке и упаковке}

%% ГОСТ 19.201-78
%% 2.4.6. В подразделе "Требования к маркировке и упаковке" в общем случае указывают требования к маркировке программного изделия, варианты и способы упаковки.

\subsection{Требования к транспортированию и хранению}

%% ГОСТ 19.201-78
%% 2.4.7. В подразделе "Требования к транспортированию и хранению" должны быть указаны для программного изделия условия транспортирования, места хранения, условия хранения, условия складирования, сроки хранения в различных условиях.

\subsection{Специальные требования}

\section{Требования к программной документации}

%% ГОСТ 19.201-78
%% 2.5а. В разделе "Требования к программной документации" должны быть указаны предварительный состав программной документации и, при необходимости, специальные требования к ней.

\section{Технико-экономические показатели}

%% ГОСТ 19.201-78
%% 2.5. В разделе "Технико-экономические показатели" должны быть указаны: ориентировочная экономическая эффективность, предполагаемая годовая потребность, экономические преимущества разработки по сравнению с лучшими отечественными и зарубежными образцами или аналогами.

\section{Стадии и этапы разработки}

%% ГОСТ 19.201-78
%% 2.6. В разделе "Стадии и этапы разработки" устанавливают необходимые стадии разработки, этапы и содержание работ (перечень программных документов, которые должны быть разработаны, согласованы и утверждены), а также, как правило, сроки разработки и определяют исполнителей.

\section{Порядок контроля и приёмки}

%% ГОСТ 19.201-78
%% 2.7. В разделе "Порядок контроля и приемки" должны быть указаны виды испытаний и общие требования к приемке работы.

\section{Приложение А}

%% ГОСТ 19.201-78
%% 2.8. В приложениях к техническому заданию, при необходимости, приводят:
%%    — перечень научно-исследовательских и других работ, обосновывающих разработку;
%%    — схемы алгоритмов, таблицы, описания, обоснования, расчёты и другие документы, которые могут быть использованы при разработке;
%%    — другие источники разработки.

\end{document}
